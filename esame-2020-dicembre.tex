\documentclass[12pt, a4paper]{article}
\usepackage[utf8]{inputenc}
\usepackage[T1]{fontenc}
\usepackage{amsmath, amssymb, amsfonts}
\usepackage[italian]{babel}
\usepackage{geometry}
\geometry{a4paper, margin=1in}

% Ambiente per le soluzioni
\newenvironment{solution}
{\par\noindent\rule{\textwidth}{0.4pt}\par\textbf{Soluzione:}\medskip\par}
{\par\rule{\textwidth}{0.4pt}\par\bigskip}

\begin{document}

\begin{center}
\Large\textbf{Calcolo differenziale ed integrale 2}\\
\large\textbf{Prova scritta --- 9 Dicembre 2020}\\
\large\textbf{SOLUZIONI COMPLETE}
\end{center}

\vspace{1cm}

\section*{Esercizio 1}

\textbf{Testo:} Stabilire se le seguenti serie convergono semplicemente e/o assolutamente.
\begin{enumerate}
    \item $\displaystyle\sum_{n=1}^{+\infty} \cos(n\pi) \frac{\sqrt{n+1}}{2n+3}$
    \item $\displaystyle\sum_{n=1}^{+\infty} \frac{\log(n^{2}+1)}{n}$
    \item $\displaystyle\sum_{n=1}^{+\infty} \left(e^{\frac{1}{2n^{2}+3}} - 1\right)$
\end{enumerate}

\begin{solution}
\textbf{Parte a)} Studiamo la serie $\displaystyle\sum_{n=1}^{+\infty} \cos(n\pi) \frac{\sqrt{n+1}}{2n+3}$.

Osserviamo che $\cos(n\pi) = (-1)^n$, quindi la serie diventa:
\[
\sum_{n=1}^{+\infty} (-1)^n \frac{\sqrt{n+1}}{2n+3}
\]

Questa è una serie a segni alterni. Sia $a_n = \frac{\sqrt{n+1}}{2n+3}$.

Per $n$ grande: $a_n = \frac{\sqrt{n+1}}{2n+3} \sim \frac{\sqrt{n}}{2n} = \frac{1}{2\sqrt{n}}$.

Applichiamo il criterio di Leibniz:
\begin{enumerate}
    \item $a_n > 0$ per ogni $n \geq 1$ (vero)
    \item $a_n \to 0$ per $n \to +\infty$ (vero perché $a_n \sim \frac{1}{2\sqrt{n}}$)
    \item $\{a_n\}$ è decrescente (verificabile derivando)
\end{enumerate}

Quindi la serie converge semplicemente per Leibniz.

Per la convergenza assoluta, studiamo $\displaystyle\sum_{n=1}^{+\infty} \frac{\sqrt{n+1}}{2n+3}$.

Poiché $\frac{\sqrt{n+1}}{2n+3} \sim \frac{1}{2\sqrt{n}}$ e $\displaystyle\sum_{n=1}^{+\infty} \frac{1}{n^{1/2}}$ diverge, la serie non converge assolutamente.

\textbf{Conclusione:} La serie converge semplicemente ma non assolutamente.

\vspace{0.5cm}

\textbf{Parte b)} Studiamo la serie $\displaystyle\sum_{n=1}^{+\infty} \frac{\log(n^{2}+1)}{n}$.

Per $n$ grande: $\log(n^2 + 1) \sim \log(n^2) = 2\log(n)$.

Quindi: $\frac{\log(n^{2}+1)}{n} \sim \frac{2\log(n)}{n}$.

La serie $\displaystyle\sum_{n=1}^{+\infty} \frac{\log(n)}{n}$ diverge (criterio dell'integrale).

Per il criterio del confronto asintotico, la serie data diverge.

\textbf{Conclusione:} La serie diverge.

\vspace{0.5cm}

\textbf{Parte c)} Studiamo la serie $\displaystyle\sum_{n=1}^{+\infty} \left(e^{\frac{1}{2n^{2}+3}} - 1\right)$.

Per $n$ grande: $\frac{1}{2n^2 + 3} \to 0$, quindi $e^{\frac{1}{2n^{2}+3}} - 1 \sim \frac{1}{2n^2 + 3}$.

Inoltre: $\frac{1}{2n^2 + 3} \sim \frac{1}{2n^2}$.

La serie $\displaystyle\sum_{n=1}^{+\infty} \frac{1}{n^2}$ converge (serie armonica generalizzata con $\alpha = 2 > 1$).

Per il criterio del confronto asintotico, la serie data converge.

\textbf{Conclusione:} La serie converge (assolutamente).
\end{solution}

\newpage

\section*{Esercizio 2}

\textbf{Testo:} Data la funzione $f: \mathbb{R} \rightarrow \mathbb{R}$, $f(x) = x^{3} - 1 - \log(1+x^{3})$, calcolare il polinomio di Taylor $T_{9}$ di centro $x_{0}=0$ e ordine 9. Determinare inoltre $f^{(9)}(0)$.

\begin{solution}
Scomponiamo $f(x) = x^3 - 1 - \log(1+x^3)$.

Utilizziamo lo sviluppo di Taylor di $\log(1+t)$:
\[
\log(1+t) = t - \frac{t^2}{2} + \frac{t^3}{3} - \frac{t^4}{4} + \frac{t^5}{5} - \frac{t^6}{6} + \frac{t^7}{7} - \frac{t^8}{8} + \frac{t^9}{9} + O(t^{10})
\]

Sostituendo $t = x^3$:
\[
\log(1+x^3) = x^3 - \frac{(x^3)^2}{2} + \frac{(x^3)^3}{3} - \frac{(x^3)^4}{4} + O(x^{15})
\]
\[
= x^3 - \frac{x^6}{2} + \frac{x^9}{3} - \frac{x^{12}}{4} + O(x^{15})
\]

Quindi:
\[
f(x) = x^3 - 1 - \left(x^3 - \frac{x^6}{2} + \frac{x^9}{3} + O(x^{12})\right)
\]
\[
= x^3 - 1 - x^3 + \frac{x^6}{2} - \frac{x^9}{3} + O(x^{12})
\]
\[
= -1 + \frac{x^6}{2} - \frac{x^9}{3} + O(x^{12})
\]

\textbf{Polinomio di Taylor di ordine 9:}
\[
T_9(x) = -1 + \frac{x^6}{2} - \frac{x^9}{3}
\]

\textbf{Per $f^{(9)}(0)$:}

Dal polinomio di Taylor:
\[
f(x) = f(0) + f'(0)x + \frac{f''(0)}{2!}x^2 + \cdots + \frac{f^{(9)}(0)}{9!}x^9 + O(x^{10})
\]

Confrontando con $f(x) = -1 + \frac{x^6}{2} - \frac{x^9}{3} + O(x^{10})$:

Il coefficiente di $x^9$ è $-\frac{1}{3}$, quindi:
\[
\frac{f^{(9)}(0)}{9!} = -\frac{1}{3}
\]
\[
f^{(9)}(0) = -\frac{9!}{3} = -\frac{362880}{3} = -120960
\]
\end{solution}

\newpage

\section*{Esercizio 3}

\textbf{Testo:} Trovare la serie di Fourier della funzione periodica di periodo $2\pi$ definita da
\[ \begin{cases} 1 & -\pi \le x < 0 \\ 4 & 0 \le x < \pi \end{cases} \]
Dire a che valore converge la serie per $x=0$.

\begin{solution}
La funzione è definita su $[-\pi, \pi)$ e ha periodo $2\pi$.

\textbf{Calcolo dei coefficienti di Fourier:}

\[
a_0 = \frac{1}{\pi} \int_{-\pi}^{\pi} f(x) dx = \frac{1}{\pi} \left[\int_{-\pi}^{0} 1 dx + \int_{0}^{\pi} 4 dx\right]
\]
\[
= \frac{1}{\pi} [\pi + 4\pi] = \frac{5\pi}{\pi} = 5
\]

\[
a_n = \frac{1}{\pi} \int_{-\pi}^{\pi} f(x) \cos(nx) dx = \frac{1}{\pi} \left[\int_{-\pi}^{0} \cos(nx) dx + \int_{0}^{\pi} 4\cos(nx) dx\right]
\]

\[
\int_{-\pi}^{0} \cos(nx) dx = \left[\frac{\sin(nx)}{n}\right]_{-\pi}^{0} = \frac{1}{n}[0 - \sin(-n\pi)] = 0
\]

\[
\int_{0}^{\pi} \cos(nx) dx = \left[\frac{\sin(nx)}{n}\right]_{0}^{\pi} = \frac{1}{n}[\sin(n\pi) - 0] = 0
\]

Quindi $a_n = 0$ per $n \geq 1$.

\[
b_n = \frac{1}{\pi} \int_{-\pi}^{\pi} f(x) \sin(nx) dx = \frac{1}{\pi} \left[\int_{-\pi}^{0} \sin(nx) dx + \int_{0}^{\pi} 4\sin(nx) dx\right]
\]

\[
\int_{-\pi}^{0} \sin(nx) dx = \left[-\frac{\cos(nx)}{n}\right]_{-\pi}^{0} = -\frac{1}{n}[1 - \cos(-n\pi)] = -\frac{1}{n}[1 - (-1)^n]
\]

\[
\int_{0}^{\pi} \sin(nx) dx = \left[-\frac{\cos(nx)}{n}\right]_{0}^{\pi} = -\frac{1}{n}[\cos(n\pi) - 1] = -\frac{1}{n}[(-1)^n - 1]
\]

Quindi:
\[
b_n = \frac{1}{\pi} \left[-\frac{1}{n}[1 - (-1)^n] - \frac{4}{n}[(-1)^n - 1]\right]
\]
\[
= \frac{1}{\pi n} \left[-(1 - (-1)^n) - 4((-1)^n - 1)\right]
\]
\[
= \frac{1}{\pi n} \left[-1 + (-1)^n - 4(-1)^n + 4\right]
\]
\[
= \frac{1}{\pi n} \left[3 - 3(-1)^n\right] = \frac{3}{\pi n}[1 - (-1)^n]
\]

Per $n$ dispari: $b_n = \frac{3}{\pi n}[1 - (-1)] = \frac{6}{\pi n}$.
Per $n$ pari: $b_n = \frac{3}{\pi n}[1 - 1] = 0$.

\textbf{Serie di Fourier:}
\[
f(x) \sim \frac{5}{2} + \sum_{k=1}^{\infty} \frac{6}{\pi(2k-1)} \sin((2k-1)x)
\]
\[
= \frac{5}{2} + \frac{6}{\pi} \left[\sin(x) + \frac{\sin(3x)}{3} + \frac{\sin(5x)}{5} + \cdots\right]
\]

\textbf{Convergenza in $x = 0$:}

Nel punto $x = 0$, la funzione ha una discontinuità di salto:
\[
\lim_{x \to 0^-} f(x) = 1, \quad \lim_{x \to 0^+} f(x) = 4
\]

La serie converge alla media dei limiti:
\[
\frac{1 + 4}{2} = \frac{5}{2}
\]

\textbf{Conclusione:} Per $x = 0$, la serie converge a $\frac{5}{2}$.
\end{solution}

\newpage

\section*{Esercizio 4}

\textbf{Testo:} Sia $f(x,y) = 3x^{2} + 4y^{2} + 6x - 12$.
\begin{enumerate}
    \item[a)] Determinare il dominio D di f e stabilire se è aperto, connesso e/o limitato.
    \item[b)] Stabilire se f è differenziabile sul suo dominio e calcolarne la derivata nel punto $P_{0}=(-1,1)$ lungo il vettore $v=(1,2)$. Determinare poi l'equazione del piano tangente al grafico di f in $(-1,1,f(-1,1))$.
    \item[c)] Determinare i punti critici di f nell'interno del suo dominio e classificarli.
    \item[d)] Determinare, se esistono, punti di massimo e minimo assoluto di f sull'insieme
    \[ C = \{(x,y) \in \mathbb{R}^{2} | x^{2}+y^{2}/2=1\} \]
\end{enumerate}

\begin{solution}
\textbf{Punto a)} La funzione $f(x,y) = 3x^{2} + 4y^{2} + 6x - 12$ è un polinomio, quindi ha dominio $D = \mathbb{R}^2$.

L'insieme $\mathbb{R}^2$ è:
\begin{itemize}
    \item Aperto (ogni punto è interno)
    \item Non limitato (contiene punti arbitrariamente lontani dall'origine)
    \item Connesso (ogni due punti possono essere collegati da un cammino)
\end{itemize}

\vspace{0.5cm}

\textbf{Punto b)} Calcoliamo le derivate parziali:
\[
\frac{\partial f}{\partial x} = 6x + 6
\]
\[
\frac{\partial f}{\partial y} = 8y
\]

Entrambe sono continue su $\mathbb{R}^2$, quindi f è differenziabile su tutto il suo dominio.

\textbf{Derivata direzionale:}

Il gradiente in $P_0 = (-1,1)$ è:
\[
\nabla f(-1,1) = (6(-1) + 6, 8 \cdot 1) = (0, 8)
\]

Il vettore unitario nella direzione di $v = (1,2)$ è:
\[
\hat{v} = \frac{(1,2)}{\sqrt{1+4}} = \frac{(1,2)}{\sqrt{5}}
\]

La derivata direzionale è:
\[
\frac{\partial f}{\partial v}(-1,1) = \nabla f(-1,1) \cdot \hat{v} = (0,8) \cdot \frac{(1,2)}{\sqrt{5}} = \frac{16}{\sqrt{5}}
\]

\textbf{Piano tangente:}

Calcoliamo $f(-1,1) = 3 + 4 - 6 - 12 = -11$.

L'equazione del piano tangente in $(-1,1,-11)$ è:
\[
z - (-11) = 0(x - (-1)) + 8(y - 1)
\]
\[
z = 8y - 8 - 11 = 8y - 19
\]

\vspace{0.5cm}

\textbf{Punto c)} Per trovare i punti critici, risolviamo:
\[
\begin{cases}
\frac{\partial f}{\partial x} = 6x + 6 = 0 \\
\frac{\partial f}{\partial y} = 8y = 0
\end{cases}
\]

Dalla prima equazione: $x = -1$.
Dalla seconda equazione: $y = 0$.

L'unico punto critico è $(-1, 0)$.

Per classificarlo, calcoliamo la matrice Hessiana:
\[
H_f = \begin{pmatrix}
\frac{\partial^2 f}{\partial x^2} & \frac{\partial^2 f}{\partial x \partial y} \\
\frac{\partial^2 f}{\partial y \partial x} & \frac{\partial^2 f}{\partial y^2}
\end{pmatrix} = \begin{pmatrix}
6 & 0 \\
0 & 8
\end{pmatrix}
\]

$\det(H_f) = 48 > 0$ e $\text{tr}(H_f) = 14 > 0$, quindi $(-1, 0)$ è un minimo locale.

\vspace{0.5cm}

\textbf{Punto d)} Usiamo i moltiplicatori di Lagrange per ottimizzare $f$ sotto il vincolo $g(x,y) = x^2 + \frac{y^2}{2} - 1 = 0$.

Il sistema è:
\[
\begin{cases}
\nabla f = \lambda \nabla g \\
g(x,y) = 0
\end{cases}
\]

Cioè:
\[
\begin{cases}
6x + 6 = 2\lambda x \\
8y = \lambda y \\
x^2 + \frac{y^2}{2} = 1
\end{cases}
\]

Dalla seconda equazione: $y = 0$ oppure $\lambda = 8$.

\textbf{Caso 1:} $y = 0$, allora $x^2 = 1$, quindi $x = \pm 1$.

Dalla prima equazione: $6x + 6 = 2\lambda x$, quindi $\lambda = \frac{6x + 6}{2x} = \frac{3(x + 1)}{x}$.

Per $x = 1$: $\lambda = 6$, $f(1, 0) = 3 + 0 + 6 - 12 = -3$.
Per $x = -1$: $\lambda = 0$, $f(-1, 0) = 3 + 0 - 6 - 12 = -15$.

\textbf{Caso 2:} $\lambda = 8$, allora dalla prima equazione: $6x + 6 = 16x$, quindi $x = \frac{3}{5}$.

Dal vincolo: $\frac{9}{25} + \frac{y^2}{2} = 1$, quindi $y^2 = 2\left(1 - \frac{9}{25}\right) = \frac{32}{25}$, quindi $y = \pm\frac{4\sqrt{2}}{5}$.

$f\left(\frac{3}{5}, \pm\frac{4\sqrt{2}}{5}\right) = 3 \cdot \frac{9}{25} + 4 \cdot \frac{32}{25} + 6 \cdot \frac{3}{5} - 12 = \frac{27 + 128}{25} + \frac{18}{5} - 12 = \frac{155}{25} + \frac{90}{25} - \frac{300}{25} = -\frac{55}{25} = -\frac{11}{5}$

\textbf{Conclusione:}
\begin{itemize}
    \item Massimo assoluto: $-\frac{11}{5}$ nei punti $\left(\frac{3}{5}, \pm\frac{4\sqrt{2}}{5}\right)$
    \item Minimo assoluto: $-15$ nel punto $(-1, 0)$
\end{itemize}
\end{solution}

\end{document}
