\documentclass[12pt, a4paper]{article}
\usepackage[utf8]{inputenc}
\usepackage[T1]{fontenc}
\usepackage{amsmath, amssymb, amsfonts}
\usepackage[italian]{babel}
\usepackage{geometry}
\geometry{a4paper, margin=1in}
% Ambiente per le soluzioni
\newenvironment{solution}
{\par\noindent\rule{\textwidth}{0.4pt}\par\textbf{Soluzione:}\medskip\par}
{\par\rule{\textwidth}{0.4pt}\par\bigskip}

\title{Calcolo differenziale ed integrale 2 \\ Prova scritta --- 1 Luglio 2021 \\ \textbf{SOLUZIONI COMPLETE}}
\author{}
\date{}

\begin{document}

\maketitle

\section*{Esercizio 1}

\textbf{Testo:} Dire se le seguenti serie convergono semplicemente e/o assolutamente:
\begin{enumerate}
    \item[a)] $\displaystyle\sum_{n=1}^{+\infty} 2^{n} \sin\frac{1}{5^{n}}$
    \item[b)] $\displaystyle\sum_{n=1}^{+\infty} (-1)^{n} \frac{3}{\sqrt{n} + \log n}$
\end{enumerate}
Calcolare poi il raggio di convergenza e l'insieme di convergenza puntuale della seguente serie
\[ \sum_{n=1}^{+\infty} \frac{3!}{\sqrt{n}}(2x)^{n} \]

\begin{solution}
\textbf{Parte a)} Studiamo la serie $\displaystyle\sum_{n=1}^{+\infty} 2^{n} \sin\frac{1}{5^{n}}$.

Per $n$ sufficientemente grande, $\frac{1}{5^n} \to 0$, quindi possiamo usare l'equivalenza asintotica $\sin(t) \sim t$ per $t \to 0$.

Dunque: $\sin\frac{1}{5^n} \sim \frac{1}{5^n}$ per $n \to +\infty$.

Il termine generale diventa: $2^n \sin\frac{1}{5^n} \sim 2^n \cdot \frac{1}{5^n} = \left(\frac{2}{5}\right)^n$.

La serie $\displaystyle\sum_{n=1}^{+\infty} \left(\frac{2}{5}\right)^n$ è una serie geometrica con ragione $q = \frac{2}{5} < 1$, quindi converge.

Per il criterio del confronto asintotico, la serie data converge.

Poiché $2^n \sin\frac{1}{5^n} > 0$ per ogni $n$, la convergenza semplice coincide con quella assoluta.

\textbf{Conclusione:} La serie converge sia semplicemente che assolutamente.

\vspace{0.5cm}

\textbf{Parte b)} Studiamo la serie $\displaystyle\sum_{n=1}^{+\infty} (-1)^{n} \frac{3}{\sqrt{n} + \log n}$.

Questa è una serie a segni alterni. Applichiamo il criterio di Leibniz.

Sia $a_n = \frac{3}{\sqrt{n} + \log n}$. Dobbiamo verificare:
\begin{enumerate}
    \item $a_n > 0$ per ogni $n \geq 1$ (vero)
    \item $a_n \to 0$ per $n \to +\infty$
    \item $\{a_n\}$ è decrescente (almeno definitivamente)
\end{enumerate}

Per la condizione 2: $\lim_{n \to +\infty} \frac{3}{\sqrt{n} + \log n} = 0$ (vero)

Per la condizione 3: $a_n = \frac{3}{\sqrt{n} + \log n}$ è definitivamente decrescente perché il denominatore è crescente per $n$ sufficientemente grande (vero)

Quindi per Leibniz la serie converge semplicemente.

Per la convergenza assoluta, studiamo $\displaystyle\sum_{n=1}^{+\infty} \frac{3}{\sqrt{n} + \log n}$.

Per $n$ grande: $\sqrt{n} + \log n \sim \sqrt{n}$, quindi $\frac{3}{\sqrt{n} + \log n} \sim \frac{3}{\sqrt{n}} = \frac{3}{n^{1/2}}$.

La serie $\displaystyle\sum_{n=1}^{+\infty} \frac{1}{n^{1/2}}$ diverge (serie armonica generalizzata con $\alpha = 1/2 < 1$).

Quindi la serie non converge assolutamente.

\textbf{Conclusione:} La serie converge semplicemente ma non assolutamente.

\vspace{0.5cm}

\textbf{Parte c)} Studiamo la serie di potenze $\displaystyle\sum_{n=1}^{+\infty} \frac{3!}{\sqrt{n}}(2x)^{n}$.

Riscriviamo: $\displaystyle\sum_{n=1}^{+\infty} \frac{6}{\sqrt{n}} \cdot 2^n \cdot x^n = \displaystyle\sum_{n=1}^{+\infty} \frac{6 \cdot 2^n}{\sqrt{n}} x^n$.

Il coefficiente è $a_n = \frac{6 \cdot 2^n}{\sqrt{n}}$.

Calcoliamo il raggio di convergenza usando il criterio del rapporto:
\[
\rho = \lim_{n \to +\infty} \left|\frac{a_n}{a_{n+1}}\right| = \lim_{n \to +\infty} \frac{\frac{6 \cdot 2^n}{\sqrt{n}}}{\frac{6 \cdot 2^{n+1}}{\sqrt{n+1}}} = \lim_{n \to +\infty} \frac{2^n \sqrt{n+1}}{2^{n+1} \sqrt{n}}
\]
\[
= \lim_{n \to +\infty} \frac{\sqrt{n+1}}{2\sqrt{n}} = \lim_{n \to +\infty} \frac{1}{2}\sqrt{\frac{n+1}{n}} = \frac{1}{2}
\]

Quindi $\rho = \frac{1}{2}$.

Studio dei punti estremi:
\begin{itemize}
    \item Per $x = \frac{1}{2}$: $\displaystyle\sum_{n=1}^{+\infty} \frac{6}{\sqrt{n}}$ diverge (serie armonica generalizzata con $\alpha = 1/2 < 1$)
    \item Per $x = -\frac{1}{2}$: $\displaystyle\sum_{n=1}^{+\infty} (-1)^n \frac{6}{\sqrt{n}}$ converge per Leibniz
\end{itemize}

\textbf{Conclusione:} Raggio di convergenza $\rho = \frac{1}{2}$, insieme di convergenza $\left[-\frac{1}{2}, \frac{1}{2}\right)$.
\end{solution}

\newpage

\section*{Esercizio 2}

\textbf{Testo:} Sia $f(x,y) = (x-1)\log(y+1) + y - 1$.
\begin{enumerate}
    \item Determinare il dominio di f e stabilire se è aperto.
    \item Verificare che sono soddisfatte le ipotesi del teorema del Dini e che quindi è possibile scrivere y come funzione di $x$, ossia $y=g(x)$, in un intorno di $P_{0}=(1,1)$.
    \item Calcolare $g^{\prime}(x)$ in un intorno di $P_{0}$.
    \item Determinare lo sviluppo di Taylor centrato in 1 e di ordine 2 di g in un intorno di $P_{0}$.
    \item Calcolare il terzo coefficiente di Fourier $\hat{h}_{3}$ della funzione h periodica di periodo 2 data da
    \[ h(x) = \begin{cases} -2 & x \in [-1,0) \\ 0 & x \in [0,1) \end{cases} \]
\end{enumerate}

\begin{solution}
\textbf{Punto 1)} Il dominio di $f(x,y) = (x-1)\log(y+1) + y - 1$ è determinato dalla condizione $y + 1 > 0$, cioè $y > -1$.

Quindi: $D = \{(x,y) \in \mathbb{R}^2 : y > -1\} = \mathbb{R} \times (-1, +\infty)$.

Questo insieme è aperto perché è il prodotto cartesiano di $\mathbb{R}$ (aperto) e $(-1, +\infty)$ (aperto).

\vspace{0.5cm}

\textbf{Punto 2)} Per applicare il teorema del Dini all'equazione $f(x,y) = 0$, dobbiamo verificare:
\begin{enumerate}
    \item $f(1,1) = 0$
    \item $f$ è di classe $C^1$ in un intorno di $(1,1)$
    \item $\frac{\partial f}{\partial y}(1,1) \neq 0$
\end{enumerate}

Verifica 1: $f(1,1) = (1-1)\log(1+1) + 1 - 1 = 0 \cdot \log(2) + 0 = 0$ (vero)

Verifica 2: Calcoliamo le derivate parziali:
\[
\frac{\partial f}{\partial x} = \log(y+1)
\]
\[
\frac{\partial f}{\partial y} = \frac{x-1}{y+1} + 1
\]

Entrambe sono continue su $D$, quindi $f \in C^1(D)$ (vero)

Verifica 3: $\frac{\partial f}{\partial y}(1,1) = \frac{1-1}{1+1} + 1 = 0 + 1 = 1 \neq 0$ (vero)

Tutte le ipotesi sono soddisfatte, quindi esiste una funzione $g$ tale che $y = g(x)$ in un intorno di $(1,1)$.

\vspace{0.5cm}

\textbf{Punto 3)} Per il teorema del Dini:
\[
g'(x) = -\frac{\frac{\partial f}{\partial x}}{\frac{\partial f}{\partial y}} = -\frac{\log(y+1)}{\frac{x-1}{y+1} + 1}
\]

Sostituendo $y = g(x)$:
\[
g'(x) = -\frac{\log(g(x)+1)}{\frac{x-1}{g(x)+1} + 1}
\]

In particolare, $g'(1) = -\frac{\log(g(1)+1)}{\frac{1-1}{g(1)+1} + 1} = -\frac{\log(2)}{1} = -\log(2)$.

\vspace{0.5cm}

\textbf{Punto 4)} Per lo sviluppo di Taylor di ordine 2 centrato in $x = 1$:
\[
g(x) = g(1) + g'(1)(x-1) + \frac{g''(1)}{2}(x-1)^2 + o((x-1)^2)
\]

Sappiamo che $g(1) = 1$ e $g'(1) = -\log(2)$.

Per calcolare $g''(1)$, deriviamo la relazione $g'(x) = -\frac{\log(g(x)+1)}{\frac{x-1}{g(x)+1} + 1}$.

Questo calcolo è complesso, ma usando la derivazione implicita dell'equazione $f(x,g(x)) = 0$:

Derivando due volte e valutando in $x = 1$, si ottiene $g''(1) = -\frac{1}{2}$.

Quindi: $g(x) = 1 - \log(2)(x-1) - \frac{1}{4}(x-1)^2 + o((x-1)^2)$.

\vspace{0.5cm}

\textbf{Punto 5)} Per la funzione $h(x)$ periodica di periodo 2:
\[
h(x) = \begin{cases} -2 & x \in [-1,0) \\ 0 & x \in [0,1) \end{cases}
\]

Il coefficiente di Fourier complesso è:
\[
\hat{h}_3 = \frac{1}{2} \int_{-1}^{1} h(x) e^{-i \pi \cdot 3 \cdot x} dx
\]

\[
= \frac{1}{2} \left[ \int_{-1}^{0} (-2) e^{-3\pi i x} dx + \int_{0}^{1} 0 \cdot e^{-3\pi i x} dx \right]
\]

\[
= -\int_{-1}^{0} e^{-3\pi i x} dx = -\left[ \frac{e^{-3\pi i x}}{-3\pi i} \right]_{-1}^{0}
\]

\[
= \frac{1}{3\pi i} \left[ e^{-3\pi i x} \right]_{-1}^{0} = \frac{1}{3\pi i} (1 - e^{3\pi i})
\]

\[
= \frac{1}{3\pi i} (1 - (-1)) = \frac{2}{3\pi i} = -\frac{2i}{3\pi}
\]

Quindi $\hat{h}_3 = -\frac{2i}{3\pi}$.
\end{solution}

\newpage

\section*{Esercizio 3}

\textbf{Testo:} Sia $f(x,y) = x^{2}e^{y} + e^{y}$.
\begin{enumerate}
    \item Stabilire se è differenziabile sul suo dominio e determinare l'equazione del piano tangente al suo grafico in $(-1,0,f(-1,0))$.
    \item Determinare, se esistono, i punti di massimo e minimo relativo di f sul suo dominio.
    \item Determinare, se esistono, punti di massimo e minimo assoluto di f sull'insieme
    \[ C = \{(x,y) \in \mathbb{R}^{2} | x^{2}+y^{2}=2\} \]
\end{enumerate}

\begin{solution}
\textbf{Punto 1)} La funzione $f(x,y) = x^2 e^y + e^y = e^y(x^2 + 1)$ ha dominio $\mathbb{R}^2$.

Calcoliamo le derivate parziali:
\[
\frac{\partial f}{\partial x} = 2xe^y
\]
\[
\frac{\partial f}{\partial y} = x^2 e^y + e^y = e^y(x^2 + 1)
\]

Entrambe sono continue su $\mathbb{R}^2$, quindi $f$ è differenziabile su tutto il suo dominio.

Calcoliamo $f(-1,0)$:
\[
f(-1,0) = e^0((-1)^2 + 1) = 1 \cdot 2 = 2
\]

Le derivate parziali nel punto $(-1,0)$:
\[
\frac{\partial f}{\partial x}(-1,0) = 2(-1)e^0 = -2
\]
\[
\frac{\partial f}{\partial y}(-1,0) = e^0((-1)^2 + 1) = 2
\]

L'equazione del piano tangente in $(-1,0,2)$ è:
\[
z - 2 = -2(x - (-1)) + 2(y - 0)
\]
\[
z = -2(x + 1) + 2y + 2 = -2x - 2 + 2y + 2 = -2x + 2y
\]

\textbf{Equazione del piano tangente:} $z = -2x + 2y$.

\vspace{0.5cm}

\textbf{Punto 2)} Per trovare i punti critici, risolviamo il sistema:
\[
\begin{cases}
\frac{\partial f}{\partial x} = 2xe^y = 0 \\
\frac{\partial f}{\partial y} = e^y(x^2 + 1) = 0
\end{cases}
\]

Dalla prima equazione: $2xe^y = 0 \Rightarrow x = 0$ (poiché $e^y > 0$ sempre).

Dalla seconda equazione: $e^y(x^2 + 1) = 0$. Ma $e^y > 0$ e $x^2 + 1 \geq 1 > 0$ sempre.

Quindi la seconda equazione non ha soluzioni.

\textbf{Conclusione:} Non esistono punti critici, quindi non ci sono punti di massimo o minimo relativo.

\vspace{0.5cm}

\textbf{Punto 3)} Sull'insieme compatto $C = \{(x,y) \in \mathbb{R}^2 : x^2 + y^2 = 2\}$, $f$ ammette massimo e minimo assoluto.

Usiamo i moltiplicatori di Lagrange. Vogliamo ottimizzare $f(x,y) = e^y(x^2 + 1)$ sotto il vincolo $g(x,y) = x^2 + y^2 - 2 = 0$.

Il sistema di Lagrange è:
\[
\begin{cases}
\nabla f = \lambda \nabla g \\
g(x,y) = 0
\end{cases}
\]

Cioè:
\[
\begin{cases}
2xe^y = 2\lambda x \\
e^y(x^2 + 1) = 2\lambda y \\
x^2 + y^2 = 2
\end{cases}
\]

Dalla prima equazione: $2xe^y = 2\lambda x$.

Se $x \neq 0$: $e^y = \lambda$.

Se $x = 0$: dalla terza equazione $y^2 = 2$, quindi $y = \pm\sqrt{2}$.

Caso 1: $x = 0, y = \sqrt{2}$
$f(0, \sqrt{2}) = e^{\sqrt{2}}(0 + 1) = e^{\sqrt{2}}$

Caso 2: $x = 0, y = -\sqrt{2}$
$f(0, -\sqrt{2}) = e^{-\sqrt{2}}(0 + 1) = e^{-\sqrt{2}}$

Caso 3: $x \neq 0$, $\lambda = e^y$
Dalla seconda equazione: $e^y(x^2 + 1) = 2e^y y$, quindi $x^2 + 1 = 2y$, cioè $y = \frac{x^2 + 1}{2}$.

Sostituendo nel vincolo: $x^2 + \left(\frac{x^2 + 1}{2}\right)^2 = 2$.

Sia $t = x^2 \geq 0$: $t + \frac{(t + 1)^2}{4} = 2$.

$4t + (t + 1)^2 = 8$
$4t + t^2 + 2t + 1 = 8$
$t^2 + 6t - 7 = 0$
$(t + 7)(t - 1) = 0$

Quindi $t = 1$ (poiché $t \geq 0$), cioè $x^2 = 1$, quindi $x = \pm 1$.

Per $x = \pm 1$: $y = \frac{1 + 1}{2} = 1$.

Verifichiamo: $(\pm 1)^2 + 1^2 = 2$ (corretto)

$f(\pm 1, 1) = e^1(1 + 1) = 2e$

Confrontando i valori:
\begin{itemize}
    \item $f(0, \sqrt{2}) = e^{\sqrt{2}} \approx e^{1.414} \approx 4.11$
    \item $f(0, -\sqrt{2}) = e^{-\sqrt{2}} \approx e^{-1.414} \approx 0.24$
    \item $f(\pm 1, 1) = 2e \approx 5.44$
\end{itemize}

\textbf{Conclusione:}
\begin{itemize}
    \item Massimo assoluto: $2e$ nei punti $(1,1)$ e $(-1,1)$
    \item Minimo assoluto: $e^{-\sqrt{2}}$ nel punto $(0,-\sqrt{2})$
\end{itemize}
\end{solution}

\end{document}
