\documentclass[12pt, a4paper]{article}
\usepackage[utf8]{inputenc}
\usepackage[T1]{fontenc}
\usepackage{amsmath, amssymb, amsfonts}
\usepackage[italian]{babel}
\usepackage{geometry}
\geometry{a4paper, margin=1in}

\newcounter{examcounter}
\newcommand{\examtitle}[2]{%
    \stepcounter{examcounter}%
    \clearpage
    \begin{center}
        \large\bfseries
        Calcolo differenziale ed integrale 2 / Calculus 2 --- Prova scritta \\
        #1
    \end{center}
    \vspace{1em}
    \setcounter{exercisecounter}{0}
}

\newcounter{exercisecounter}[examcounter]
\newenvironment{exercise}{%
    \stepcounter{exercisecounter}%
    \par\noindent\textbf{Esercizio \theexercisecounter.}\medskip\par
    \normalfont
}{\par\bigskip}

\newenvironment{solution}{%
    \par\noindent\textbf{Soluzione.}\medskip\par
    \normalfont
}{\par\bigskip}

\begin{document}

\examtitle{20 Giugno 2023}

\begin{exercise}
Stabilire se le seguenti serie sono a termini positivi e convergono semplicemente e/o assolutamente.
\begin{enumerate}
    \item[a)] $\displaystyle\sum_{n=1}^{+\infty} (-1)^{n+1} \sin\left(\frac{1001}{\sqrt{n}}\right)$
    \item[b)] $\displaystyle\sum_{n=1}^{+\infty} \frac{n(3-(\cos n)^{2})}{n^{3}+2n+1}$
    \item[c)] Data la serie di potenze $\displaystyle\sum_{n=1}^{+\infty} \frac{2^{n}+3^{-n}}{n^{2}}x^{n}$, determinarne il raggio di convergenza $\rho$ e l'insieme di convergenza puntuale I.
\end{enumerate}
\end{exercise}
\begin{solution}
\begin{enumerate}
    \item[a)] \textbf{Analisi della serie:} $\displaystyle\sum_{n=1}^{+\infty} (-1)^{n+1} \sin\left(\frac{1001}{\sqrt{n}}\right)$.
    
    Si tratta di una serie a segni alterni. Poniamo $a_n = \sin\left(\frac{1001}{\sqrt{n}}\right)$.
    
    \textbf{Convergenza semplice - Criterio di Leibniz:}
    Verifichiamo le condizioni:
    \begin{align}
        \text{1) } &\lim_{n\to+\infty} a_n = \lim_{n\to+\infty} \sin\left(\frac{1001}{\sqrt{n}}\right) = \sin(0) = 0 \\
        \text{2) } &a_n \text{ è decrescente per } n \text{ sufficientemente grande}
    \end{align}
    
    Per la condizione 2): poiché $\sin x$ è crescente su $[0, \pi/2]$ e $\frac{1001}{\sqrt{n}} \to 0^+$, per $n$ sufficientemente grande $\frac{1001}{\sqrt{n}} \in (0, \pi/2)$. Essendo $\frac{1001}{\sqrt{n}}$ decrescente, anche $\sin\left(\frac{1001}{\sqrt{n}}\right)$ è decrescente.
    
    $\Rightarrow$ \textbf{La serie converge semplicemente} per il criterio di Leibniz.
    
    \textbf{Convergenza assoluta:}
    Consideriamo:
    \[\sum_{n=1}^{+\infty} \left|(-1)^{n+1} \sin\left(\frac{1001}{\sqrt{n}}\right)\right| = \sum_{n=1}^{+\infty} \sin\left(\frac{1001}{\sqrt{n}}\right)\]
    
    Usando l'equivalenza asintotica $\sin x \sim x$ per $x \to 0$:
    \[\sin\left(\frac{1001}{\sqrt{n}}\right) \sim \frac{1001}{\sqrt{n}} \text{ per } n \to +\infty\]
    
    La serie $\displaystyle\sum_{n=1}^{+\infty} \frac{1001}{\sqrt{n}} = 1001 \sum_{n=1}^{+\infty} \frac{1}{n^{1/2}}$ è una serie armonica generalizzata con $p = \frac{1}{2} \leq 1$, quindi \textbf{diverge}.
    
    Per il criterio del confronto asintotico, la serie dei valori assoluti diverge.
    
    \textbf{Conclusione:} La serie converge semplicemente ma \textbf{non assolutamente}.

    \item[b)] \textbf{Analisi della serie:} $\displaystyle\sum_{n=1}^{+\infty} \frac{n(3-(\cos n)^{2})}{n^{3}+2n+1}$.
    
    \textbf{Serie a termini positivi:}
    Osserviamo che $0 \leq (\cos n)^2 \leq 1$, quindi:
    \[2 \leq 3-(\cos n)^2 \leq 3\]
    
    Il numeratore $n(3-(\cos n)^2) > 0$ per $n \geq 1$ e il denominatore $n^3+2n+1 > 0$ per $n \geq 1$.
    
    $\Rightarrow$ \textbf{La serie è a termini positivi}.
    
    \textbf{Convergenza - Criterio del confronto asintotico:}
    Sia $a_n = \frac{n(3-(\cos n)^{2})}{n^{3}+2n+1}$.
    
    Per $n \to +\infty$:
    \[a_n \sim \frac{n \cdot C}{n^3} = \frac{C}{n^2}\]
    dove $C$ è una costante con $2 \leq C \leq 3$.
    
    La serie di confronto:
    \[\sum_{n=1}^{+\infty} \frac{C}{n^2} = C \sum_{n=1}^{+\infty} \frac{1}{n^2}\]
    è una serie armonica generalizzata con $p = 2 > 1$, quindi \textbf{converge}.
    
    Per il criterio del confronto asintotico, la serie data converge.
    
    \textbf{Conclusione:} Essendo una serie a termini positivi convergente, converge anche \textbf{assolutamente}.

    \item[c)] \textbf{Serie di potenze:} $\displaystyle\sum_{n=1}^{+\infty} \frac{2^{n}+3^{-n}}{n^{2}}x^{n}$.
    
    \textbf{Calcolo del raggio di convergenza:}
    Sia $c_n = \frac{2^{n}+3^{-n}}{n^{2}}$. Utilizziamo il criterio del rapporto:
    
    \begin{align}
        L &= \lim_{n\to+\infty} \left|\frac{c_{n+1}}{c_n}\right| \\
        &= \lim_{n\to+\infty} \frac{2^{n+1}+3^{-(n+1)}}{(n+1)^2} \cdot \frac{n^2}{2^n+3^{-n}} \\
        &= \lim_{n\to+\infty} \frac{2^{n+1}(1+3^{-(n+1)}/2^{n+1})}{2^n(1+3^{-n}/2^n)} \cdot \frac{n^2}{(n+1)^2} \\
        &= \lim_{n\to+\infty} 2 \cdot \frac{1+(1/3)(1/6)^{n+1}}{1+(1/6)^n} \cdot \left(\frac{n}{n+1}\right)^2 \\
        &= 2 \cdot 1 \cdot 1 = 2
    \end{align}
    
    Il raggio di convergenza è $\rho = \frac{1}{L} = \frac{1}{2}$.
    
    \textbf{Studio degli estremi dell'intervallo:}
    La serie converge per $|x| < \frac{1}{2}$, cioè $x \in \left(-\frac{1}{2}, \frac{1}{2}\right)$.
    
    \begin{itemize}
        \item \textbf{Per $x = \frac{1}{2}$:}
        \[\sum_{n=1}^{+\infty} \frac{2^{n}+3^{-n}}{n^{2}} \left(\frac{1}{2}\right)^n = \sum_{n=1}^{+\infty} \frac{1+(1/6)^n}{n^2}\]
        
        Il termine generale è asintotico a $\frac{1}{n^2}$. Poiché $\sum \frac{1}{n^2}$ converge, la serie converge per $x = \frac{1}{2}$.
        
        \item \textbf{Per $x = -\frac{1}{2}$:}
        \[\sum_{n=1}^{+\infty} \frac{2^{n}+3^{-n}}{n^{2}} \left(-\frac{1}{2}\right)^n = \sum_{n=1}^{+\infty} (-1)^n \frac{1+(1/6)^n}{n^2}\]
        
        Per la convergenza assoluta:
        \[\sum_{n=1}^{+\infty} \left|(-1)^n \frac{1+(1/6)^n}{n^2}\right| = \sum_{n=1}^{+\infty} \frac{1+(1/6)^n}{n^2}\]
        che converge (come dimostrato sopra).
        
        Quindi la serie converge assolutamente per $x = -\frac{1}{2}$.
    \end{itemize}
    
    \textbf{Conclusione:} L'insieme di convergenza puntuale è $I = \left[-\frac{1}{2}, \frac{1}{2}\right]$.
\end{enumerate}
\end{solution}

\begin{exercise}
\begin{enumerate}
    \item[a)] Data la funzione $f(x) = x(e^{x}-1) + \ln(1+x^{2}) + \sin(2x)$, calcolarne il polinomio di Me Laurin di ordine 4.
    \item[b)] Scrivere il resto di Lagrange $R_{1}(x)$ di ordine 1 (con centro in 0) della funzione $g(x) = \cos(x^{2}) + \sin(3x)$ e determinarne una stima per $x \in (0, 1/4]$.
\end{enumerate}
\end{exercise}
\begin{solution}
\begin{enumerate}
    \item[a)] \textbf{Polinomio di MacLaurin di ordine 4}
    
    Data $f(x) = x(e^{x}-1) + \ln(1+x^{2}) + \sin(2x)$.
    
    Calcoliamo gli sviluppi di MacLaurin delle singole funzioni:
    
    \begin{align}
        e^x &= 1 + x + \frac{x^2}{2!} + \frac{x^3}{3!} + \frac{x^4}{4!} + o(x^4) \\
        e^x - 1 &= x + \frac{x^2}{2} + \frac{x^3}{6} + \frac{x^4}{24} + o(x^4) \\
        x(e^x - 1) &= x^2 + \frac{x^3}{2} + \frac{x^4}{6} + o(x^4)
    \end{align}
    
    Per il logaritmo, usando $\ln(1+t) = t - \frac{t^2}{2} + \frac{t^3}{3} + o(t^3)$ con $t = x^2$:
    \[\ln(1+x^2) = x^2 - \frac{x^4}{2} + o(x^4)\]
    
    Per il seno, usando $\sin(u) = u - \frac{u^3}{3!} + o(u^4)$ con $u = 2x$:
    \[\sin(2x) = 2x - \frac{8x^3}{6} + o(x^4) = 2x - \frac{4x^3}{3} + o(x^4)\]
    
    \textbf{Sommando tutti i termini:}
    \begin{align}
        f(x) &= \left(x^2 + \frac{x^3}{2} + \frac{x^4}{6}\right) + \left(x^2 - \frac{x^4}{2}\right) + \left(2x - \frac{4x^3}{3}\right) + o(x^4) \\
        &= 2x + 2x^2 + \left(\frac{1}{2} - \frac{4}{3}\right)x^3 + \left(\frac{1}{6} - \frac{1}{2}\right)x^4 + o(x^4) \\
        &= 2x + 2x^2 - \frac{5}{6}x^3 - \frac{1}{3}x^4 + o(x^4)
    \end{align}
    
    \textbf{Risultato:} $P_4(x) = 2x + 2x^2 - \frac{5}{6}x^3 - \frac{1}{3}x^4$.

    \item[b)] \textbf{Resto di Lagrange di ordine 1}
    
    Data $g(x) = \cos(x^2) + \sin(3x)$.
    
    \textbf{Polinomio di MacLaurin di ordine 1:}
    Il polinomio è $P_1(x) = g(0) + g'(0)x$.
    
    Calcoliamo:
    \begin{align}
        g(0) &= \cos(0) + \sin(0) = 1 + 0 = 1 \\
        g'(x) &= -\sin(x^2) \cdot 2x + \cos(3x) \cdot 3 = -2x\sin(x^2) + 3\cos(3x) \\
        g'(0) &= -2(0)\sin(0) + 3\cos(0) = 0 + 3 = 3
    \end{align}
    
    Quindi $P_1(x) = 1 + 3x$.
    
    \textbf{Resto di Lagrange:}
    \[R_1(x) = \frac{g''(\xi)}{2!}x^2 \text{ con } \xi \in (0,x)\]
    
    Calcoliamo $g''(x)$:
    \begin{align}
        g''(x) &= \frac{d}{dx}(-2x\sin(x^2) + 3\cos(3x)) \\
        &= -2\sin(x^2) - 2x\cos(x^2) \cdot 2x - 3\sin(3x) \cdot 3 \\
        &= -2\sin(x^2) - 4x^2\cos(x^2) - 9\sin(3x)
    \end{align}
    
    Quindi:
    \[R_1(x) = \frac{-2\sin(\xi^2) - 4\xi^2\cos(\xi^2) - 9\sin(3\xi)}{2}x^2\]
    
    \textbf{Stima per $x \in (0, 1/4]$:}
    
    Per $\xi \in (0, x] \subseteq (0, 1/4]$, abbiamo $\xi^2 \in (0, 1/16]$ e $3\xi \in (0, 3/4]$.
    
    Tutti gli argomenti delle funzioni trigonometriche sono positivi e minori di $\pi/2$, quindi:
    $\sin(\xi^2) > 0$, $\cos(\xi^2) > 0$, $\sin(3\xi) > 0$.
    
    Il numeratore $N(\xi) = -2\sin(\xi^2) - 4\xi^2\cos(\xi^2) - 9\sin(3\xi)$ è negativo.
    
    \begin{align}
        |R_1(x)| &= \frac{2\sin(\xi^2) + 4\xi^2\cos(\xi^2) + 9\sin(3\xi)}{2}x^2
    \end{align}
    
    Usando le maggiorazioni $\sin u \leq u$ per $u \geq 0$ e $\cos u \leq 1$:
    \begin{align}
        2\sin(\xi^2) + 4\xi^2\cos(\xi^2) + 9\sin(3\xi) &\leq 2\xi^2 + 4\xi^2 + 9(3\xi) \\
        &= 6\xi^2 + 27\xi \\
        &\leq 6x^2 + 27x
    \end{align}
    
    Per $x \in (0, 1/4]$:
    \[6x^2 + 27x \leq 6 \cdot \frac{1}{16} + 27 \cdot \frac{1}{4} = \frac{3}{8} + \frac{27}{4} = \frac{57}{8}\]
    
    \textbf{Stima finale:}
    \[|R_1(x)| \leq \frac{57/8}{2}x^2 = \frac{57}{16}x^2\]
\end{enumerate}
\end{solution}
\begin{exercise}[Serie di Fourier]
Sia $f$ la funzione ottenuta estendendo per periodicità a tutto $\mathbb{R}$ la funzione
\[ g(x) = \begin{cases} 2\pi+x & x \in [-\pi,0) \\ 2\pi-x & x \in [0,\pi) \end{cases} \]
Si chiede di:
\begin{enumerate}
    \item Calcolare il coefficiente di Fourier $\hat{f}_{0}$.
    \item Calcolare il coefficiente di Fourier $\hat{f}_{k}$ per $k \ne 0$.
    \item Calcolare il valore della serie di Fourier di f sull'intervallo $[-\pi, \pi)$.
\end{enumerate}
\end{exercise}
\begin{solution}
\begin{enumerate}
    \item[a)] \textbf{Coefficiente $\hat{f}_0$}
    
    Il coefficiente $\hat{f}_0$ è il valore medio della funzione su un periodo $T = 2\pi$:
    \[\hat{f}_0 = \frac{1}{2\pi} \int_{-\pi}^{\pi} g(x) \, dx\]
    
    \textbf{Osservazione:} La funzione $g(x)$ è pari.
    
    Infatti: Per $x \in [-\pi, 0)$, $g(-x) = 2\pi - x = g(x)$ dove $-x \in (0, \pi]$.
    Per $x \in [0, \pi)$, $g(-x) = 2\pi + (-x) = 2\pi - x = g(x)$ dove $-x \in (-\pi, 0]$.
    
    Quindi: $\int_{-\pi}^{\pi} g(x) \, dx = 2 \int_{0}^{\pi} g(x) \, dx$.
    
    \begin{align}
        \hat{f}_0 &= \frac{1}{2\pi} \cdot 2 \int_{0}^{\pi} (2\pi-x) \, dx \\
        &= \frac{1}{\pi} \left[ 2\pi x - \frac{x^2}{2} \right]_0^{\pi} \\
        &= \frac{1}{\pi} \left( 2\pi^2 - \frac{\pi^2}{2} \right) \\
        &= \frac{1}{\pi} \cdot \frac{4\pi^2 - \pi^2}{2} = \frac{1}{\pi} \cdot \frac{3\pi^2}{2} = \frac{3\pi}{2}
    \end{align}
    
    \textbf{Nota:} Questo è il coefficiente $\hat{f}_0$ in forma complessa. Il corrispondente $a_0 = 2\hat{f}_0 = 3\pi$.

    \item[b)] \textbf{Coefficienti $\hat{f}_k$ per $k \neq 0$}
    
    Utilizziamo la forma reale dei coefficienti di Fourier:
    \begin{align}
        a_k &= \frac{1}{\pi} \int_{-\pi}^{\pi} g(x) \cos(kx) \, dx \\
        b_k &= \frac{1}{\pi} \int_{-\pi}^{\pi} g(x) \sin(kx) \, dx
    \end{align}
    
    Poiché $g(x)$ è pari:
    \begin{itemize}
        \item $g(x)\sin(kx)$ è dispari $\Rightarrow b_k = 0$ per ogni $k$
        \item $g(x)\cos(kx)$ è pari $\Rightarrow a_k = \frac{2}{\pi} \int_{0}^{\pi} g(x) \cos(kx) \, dx$
    \end{itemize}
    
    \textbf{Calcolo di $a_k$:}
    \[a_k = \frac{2}{\pi} \int_{0}^{\pi} (2\pi-x) \cos(kx) \, dx\]
    
    Integrazione per parti con $u = 2\pi-x$, $du = -dx$, $dv = \cos(kx) \, dx$, $v = \frac{\sin(kx)}{k}$:
    
    \begin{align}
        a_k &= \frac{2}{\pi} \left[ (2\pi-x)\frac{\sin(kx)}{k} \right]_0^{\pi} - \frac{2}{\pi} \int_{0}^{\pi} \frac{\sin(kx)}{k} \cdot (-dx) \\
        &= \frac{2}{\pi} \left[ \pi \cdot \frac{\sin(k\pi)}{k} - 2\pi \cdot \frac{\sin(0)}{k} \right] + \frac{2}{\pi k} \int_{0}^{\pi} \sin(kx) \, dx
    \end{align}
    
    Poiché $\sin(k\pi) = 0$ e $\sin(0) = 0$ per $k$ intero, il primo termine è nullo.
    
    \begin{align}
        a_k &= \frac{2}{\pi k} \left[ -\frac{\cos(kx)}{k} \right]_0^{\pi} \\
        &= \frac{2}{\pi k^2} \left[ -\cos(k\pi) + \cos(0) \right] \\
        &= \frac{2}{\pi k^2} (1 - \cos(k\pi))
    \end{align}
    
    Usando $\cos(k\pi) = (-1)^k$:
    \[a_k = \frac{2(1-(-1)^k)}{\pi k^2}\]
    
    \textbf{Risultato:}
    \begin{itemize}
        \item Se $k$ è pari: $a_k = 0$
        \item Se $k$ è dispari: $a_k = \frac{4}{\pi k^2}$
    \end{itemize}
    
    I coefficienti complessi sono $\hat{f}_k = \frac{a_k}{2}$ (poiché $b_k = 0$):
    
    \[\hat{f}_k = \begin{cases}
        0 & \text{se } k \text{ è pari e } k \neq 0 \\
        \frac{2}{\pi k^2} & \text{se } k \text{ è dispari}
    \end{cases}\]

    \item[c)] \textbf{Valore della serie di Fourier}
    
    La serie di Fourier di $f(x)$ in forma reale è:
    \[S_f(x) = \frac{a_0}{2} + \sum_{k=1}^{\infty} a_k \cos(kx)\]
    
    Sostituendo i valori calcolati:
    \[S_f(x) = \frac{3\pi}{2} + \sum_{m=1}^{\infty} \frac{4}{\pi (2m-1)^2} \cos((2m-1)x)\]
    
    \textbf{Studio della continuità:}
    
    La funzione $g(x) = 2\pi - |x|$ per $x \in [-\pi, \pi)$ ha i seguenti valori agli estremi:
    \begin{align}
        g(0) &= 2\pi \\
        \lim_{x \to \pi^-} g(x) &= \lim_{x \to \pi^-} (2\pi-x) = \pi \\
        \lim_{x \to -\pi^+} g(x) &= \lim_{x \to -\pi^+} (2\pi+x) = \pi
    \end{align}
    
    Per periodicità, $f(\pi) = f(-\pi)$, quindi la funzione estesa $f(x)$ è continua su tutto $\mathbb{R}$.
    
    \textbf{Convergenza della serie:}
    
    Per il teorema di convergenza puntuale di Dirichlet, se $f$ è continua in $x$, allora $S_f(x) = f(x)$.
    
    Dato che $f(x)$ è continua ovunque, $S_f(x) = f(x)$ per ogni $x \in \mathbb{R}$.
    
    \textbf{Risultato sull'intervallo $[-\pi, \pi)$:}
    
    \[S_f(x) = g(x) = \begin{cases} 
        2\pi+x & \text{se } x \in [-\pi,0) \\ 
        2\pi-x & \text{se } x \in [0,\pi) 
    \end{cases}\]
    
    In particolare:
    \begin{itemize}
        \item $S_f(-\pi) = \pi$
        \item $S_f(0) = 2\pi$
        \item La serie converge a $f(\pi) = \pi$
    \end{itemize}
    
    \textbf{Verifica del calcolo di $S_f(-\pi) = \pi$ usando la serie:}
    
    Sostituendo $x = -\pi$ nella serie:
    \begin{align}
        S_f(-\pi) &= \frac{3\pi}{2} + \sum_{m=1}^{\infty} \frac{4}{\pi (2m-1)^2} \cos((2m-1)(-\pi)) \\
        &= \frac{3\pi}{2} + \sum_{m=1}^{\infty} \frac{4}{\pi (2m-1)^2} \cos((2m-1)\pi)
    \end{align}
    
    Poiché $\cos((2m-1)\pi) = -1$ per ogni $m \geq 1$ (argomento dispari multiplo di $\pi$):
    \begin{align}
        S_f(-\pi) &= \frac{3\pi}{2} - \frac{4}{\pi} \sum_{m=1}^{\infty} \frac{1}{(2m-1)^2}
    \end{align}
    
    Per calcolare $\sum_{m=1}^{\infty} \frac{1}{(2m-1)^2}$, usiamo l'identità di Eulero e la scomposizione in termini pari e dispari.
    
    Dalla serie di Eulero $\sum_{n=1}^{\infty} \frac{1}{n^2} = \frac{\pi^2}{6}$, separando termini pari e dispari:
    \[\frac{\pi^2}{6} = \sum_{m=1}^{\infty} \frac{1}{(2m)^2} + \sum_{m=1}^{\infty} \frac{1}{(2m-1)^2} = \frac{1}{4}\sum_{m=1}^{\infty} \frac{1}{m^2} + \sum_{m=1}^{\infty} \frac{1}{(2m-1)^2}\]
    
    Quindi: $\sum_{m=1}^{\infty} \frac{1}{(2m-1)^2} = \frac{\pi^2}{6} - \frac{1}{4} \cdot \frac{\pi^2}{6} = \frac{\pi^2}{6} - \frac{\pi^2}{24} = \frac{\pi^2}{8}$.
    
    Sostituendo questa identità:
    \begin{align}
        S_f(-\pi) &= \frac{3\pi}{2} - \frac{4}{\pi} \cdot \frac{\pi^2}{8} = \frac{3\pi}{2} - \frac{\pi}{2} = \pi
    \end{align}
\end{enumerate}
\end{solution}

\begin{exercise}
Sia $f(x,y) = \frac{3}{2}x^{2} - 8x - 4xy + 4y^{2} + 12 \ln x$.
\begin{enumerate}
    \item Determinare il dominio di f e dire dove è differenziabile.
    \item Calcolare il gradiente nel punto $(1,0)$, la derivata direzionale $\frac{\partial f(1,0)}{\partial v}$ per $v=(-\frac{4}{5},\frac{3}{5})$ e scrivere l'equazione del piano tangente al grafico di f in $(1,0,f(1,0))$.
    \item Stabilire quali sono i punti critici di f sul suo dominio e classificarli.
\end{enumerate}
\end{exercise}
\begin{solution}
\begin{enumerate}
    \item[a)] \textbf{Dominio e differenziabilità}
    
    La funzione $f(x,y) = \frac{3}{2}x^{2} - 8x - 4xy + 4y^{2} + 12 \ln x$ contiene il termine $12 \ln x$.
    
    Per l'esistenza del logaritmo naturale: $x > 0$.
    
    \textbf{Dominio:} $D_f = \{ (x,y) \in \mathbb{R}^2 : x > 0 \}$.
    
    La funzione è somma e composizione di funzioni differenziabili nel loro dominio (polinomi e logaritmo), quindi è \textbf{differenziabile in tutto $D_f$}.

    \item[b)] \textbf{Gradiente, derivata direzionale e piano tangente}
    
    \textbf{Derivate parziali:}
    \begin{align}
        \frac{\partial f}{\partial x} &= 3x - 8 - 4y + \frac{12}{x} \\
        \frac{\partial f}{\partial y} &= -4x + 8y
    \end{align}
    
    \textbf{Gradiente nel punto $(1,0)$:}
    \begin{align}
        \nabla f(1,0) &= \left( 3(1) - 8 - 4(0) + \frac{12}{1}, -4(1) + 8(0) \right) \\
        &= (3 - 8 + 12, -4) = (7, -4)
    \end{align}
    
    \textbf{Derivata direzionale lungo $v = \left(-\frac{4}{5}, \frac{3}{5}\right)$:}
    \begin{align}
        \frac{\partial f(1,0)}{\partial v} &= \nabla f(1,0) \cdot v \\
        &= (7, -4) \cdot \left(-\frac{4}{5}, \frac{3}{5}\right) \\
        &= 7 \cdot \left(-\frac{4}{5}\right) + (-4) \cdot \frac{3}{5} \\
        &= -\frac{28}{5} - \frac{12}{5} = -8
    \end{align}
    
    \textbf{Valore della funzione in $(1,0)$:}
    \begin{align}
        f(1,0) &= \frac{3}{2}(1)^2 - 8(1) - 4(1)(0) + 4(0)^2 + 12 \ln(1) \\
        &= \frac{3}{2} - 8 + 0 + 0 + 0 = -\frac{13}{2}
    \end{align}
    
    \textbf{Equazione del piano tangente nel punto $(1,0,f(1,0))$:}
    \begin{align}
        z &= f(1,0) + \frac{\partial f}{\partial x}(1,0)(x-1) + \frac{\partial f}{\partial y}(1,0)(y-0) \\
        z &= -\frac{13}{2} + 7(x-1) - 4y \\
        z &= -\frac{13}{2} + 7x - 7 - 4y \\
        z &= 7x - 4y - \frac{27}{2}
    \end{align}
    
    Forma implicita: $14x - 8y - 2z - 27 = 0$.

    \item[c)] \textbf{Punti critici e loro classificazione}
    
    \textbf{Ricerca dei punti critici:}
    
    Poniamo $\nabla f = \mathbf{0}$:
    \[\begin{cases} 
    3x - 8 - 4y + \frac{12}{x} = 0 \\ 
    -4x + 8y = 0 
    \end{cases}\]
    
    Dalla seconda equazione: $8y = 4x \Rightarrow y = \frac{x}{2}$.
    
    Sostituendo nella prima:
    \begin{align}
        3x - 8 - 4 \cdot \frac{x}{2} + \frac{12}{x} &= 0 \\
        3x - 8 - 2x + \frac{12}{x} &= 0 \\
        x - 8 + \frac{12}{x} &= 0
    \end{align}
    
    Moltiplicando per $x > 0$:
    \[x^2 - 8x + 12 = 0\]
    
    Risolvendo: $(x-2)(x-6) = 0 \Rightarrow x_1 = 2, x_2 = 6$.
    
    \textbf{Punti critici:}
    \begin{itemize}
        \item $P_1 = (2, 1)$ con $y_1 = \frac{2}{2} = 1$
        \item $P_2 = (6, 3)$ con $y_2 = \frac{6}{2} = 3$
    \end{itemize}
    
    \textbf{Classificazione tramite matrice Hessiana:}
    
    Derivate seconde:
    \begin{align}
        \frac{\partial^2 f}{\partial x^2} &= 3 - \frac{12}{x^2} \\
        \frac{\partial^2 f}{\partial y^2} &= 8 \\
        \frac{\partial^2 f}{\partial x \partial y} &= -4
    \end{align}
    
    Matrice Hessiana:
    \[H_f(x,y) = \begin{pmatrix} 
    3 - \frac{12}{x^2} & -4 \\ 
    -4 & 8 
    \end{pmatrix}\]
    
    Determinante: $\det(H_f) = 8\left(3 - \frac{12}{x^2}\right) - 16 = 24 - \frac{96}{x^2} - 16 = 8 - \frac{96}{x^2}$.
    
    \textbf{Per $P_1 = (2,1)$:}
    \begin{align}
        \frac{\partial^2 f}{\partial x^2}(2,1) &= 3 - \frac{12}{4} = 0 \\
        \det(H_f(2,1)) &= 8 - \frac{96}{4} = 8 - 24 = -16 < 0
    \end{align}
    
    Poiché $\det(H_f(2,1)) < 0$, il punto $P_1 = (2,1)$ è un \textbf{punto di sella}.
    
    \textbf{Per $P_2 = (6,3)$:}
    \begin{align}
        \frac{\partial^2 f}{\partial x^2}(6,3) &= 3 - \frac{12}{36} = 3 - \frac{1}{3} = \frac{8}{3} > 0 \\
        \det(H_f(6,3)) &= 8 - \frac{96}{36} = 8 - \frac{8}{3} = \frac{16}{3} > 0
    \end{align}
    
    Poiché $\det(H_f(6,3)) > 0$ e $\frac{\partial^2 f}{\partial x^2}(6,3) > 0$, il punto $P_2 = (6,3)$ è un \textbf{punto di minimo locale}.
\end{enumerate}
\end{solution}

\end{document}
