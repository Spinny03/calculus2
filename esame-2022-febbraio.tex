\documentclass[12pt, a4paper]{article}
\usepackage[utf8]{inputenc}
\usepackage[T1]{fontenc}
\usepackage{amsmath, amssymb, amsfonts}
\usepackage[italian]{babel}
\usepackage{geometry}
\geometry{a4paper, margin=1in}

% Ambiente per le soluzioni
\newenvironment{solution}
{\par\noindent\rule{\textwidth}{0.4pt}\par\textbf{Soluzione:}\medskip\par}
{\par\rule{\textwidth}{0.4pt}\par\bigskip}

\begin{document}

\begin{center}
\Large\textbf{Calcolo differenziale ed integrale 2}\\
\large\textbf{Prova scritta --- 3 Febbraio 2022}\\
\large\textbf{SOLUZIONI COMPLETE}
\end{center}

\vspace{1cm}

\section*{Esercizio 1}

\textbf{Testo:} Stabilire se le seguenti serie convergono, divergono o sono indeterminate e, nel caso convergano, se c'è anche convergenza assoluta.
\begin{enumerate}
    \item[a)] $\displaystyle\sum_{n=1}^{+\infty} (-1)^{n} n^{4} \sin^{2}(1/n^{3})$
    \item[b)] $\displaystyle\sum_{n=1}^{+\infty} \frac{\cos n + 5 + ne^{n}}{n^{2}e^{n}+e^{n}}$
\end{enumerate}
Data la serie di potenze $\displaystyle\sum_{n=0}^{+\infty} \frac{x^{n}}{(n^{2}+1)2^{n}}$, determinarne il raggio di convergenza $\rho$, l'insieme I di convergenza puntuale e la serie derivata.

\begin{solution}
\textbf{Parte a)} Studiamo la serie $\displaystyle\sum_{n=1}^{+\infty} (-1)^{n} n^{4} \sin^{2}(1/n^{3})$.

Per $n$ grande, $\frac{1}{n^3} \to 0$, quindi $\sin(1/n^3) \sim \frac{1}{n^3}$.

Dunque: $\sin^2(1/n^3) \sim \frac{1}{n^6}$.

Il termine generale diventa: $(-1)^n n^4 \sin^2(1/n^3) \sim (-1)^n n^4 \cdot \frac{1}{n^6} = \frac{(-1)^n}{n^2}$.

La serie $\displaystyle\sum_{n=1}^{+\infty} \frac{(-1)^n}{n^2}$ converge assolutamente (serie armonica generalizzata con $\alpha = 2 > 1$).

Per il criterio del confronto asintotico, la serie data converge assolutamente.

\textbf{Conclusione:} La serie converge assolutamente.

\vspace{0.5cm}

\textbf{Parte b)} Studiamo la serie $\displaystyle\sum_{n=1}^{+\infty} \frac{\cos n + 5 + ne^{n}}{n^{2}e^{n}+e^{n}}$.

Semplifichiamo il denominatore: $n^{2}e^{n}+e^{n} = e^n(n^2 + 1)$.

Il termine generale diventa:
\[
\frac{\cos n + 5 + ne^{n}}{e^n(n^2 + 1)} = \frac{\cos n + 5}{e^n(n^2 + 1)} + \frac{ne^{n}}{e^n(n^2 + 1)} = \frac{\cos n + 5}{e^n(n^2 + 1)} + \frac{n}{n^2 + 1}
\]

Il primo termine: $\frac{\cos n + 5}{e^n(n^2 + 1)} \leq \frac{6}{e^n(n^2 + 1)}$, che converge perché $e^n$ cresce più velocemente di qualsiasi polinomio.

Il secondo termine: $\frac{n}{n^2 + 1} \sim \frac{1}{n}$ per $n$ grande, quindi $\displaystyle\sum_{n=1}^{+\infty} \frac{n}{n^2 + 1}$ diverge.

Quindi la serie data diverge.

\textbf{Conclusione:} La serie diverge.

\vspace{0.5cm}

\textbf{Parte c)} Studiamo la serie di potenze $\displaystyle\sum_{n=0}^{+\infty} \frac{x^{n}}{(n^{2}+1)2^{n}}$.

Il coefficiente è $a_n = \frac{1}{(n^{2}+1)2^{n}}$.

Calcoliamo il raggio di convergenza usando il criterio del rapporto:
\[
\rho = \lim_{n \to +\infty} \left|\frac{a_n}{a_{n+1}}\right| = \lim_{n \to +\infty} \frac{\frac{1}{(n^{2}+1)2^{n}}}{\frac{1}{((n+1)^{2}+1)2^{n+1}}}
\]
\[
= \lim_{n \to +\infty} \frac{((n+1)^{2}+1)2^{n+1}}{(n^{2}+1)2^{n}} = \lim_{n \to +\infty} \frac{2((n+1)^{2}+1)}{n^{2}+1} = 2
\]

Quindi $\rho = 2$.

Studio dei punti estremi:
\begin{itemize}
    \item Per $x = 2$: $\displaystyle\sum_{n=0}^{+\infty} \frac{1}{n^{2}+1}$ converge
    \item Per $x = -2$: $\displaystyle\sum_{n=0}^{+\infty} \frac{(-1)^n}{n^{2}+1}$ converge assolutamente
\end{itemize}

\textbf{Insieme di convergenza:} $[-2, 2]$.

\textbf{Serie derivata:}
\[
\sum_{n=1}^{+\infty} \frac{nx^{n-1}}{(n^{2}+1)2^{n}}
\]

\textbf{Conclusione:} $\rho = 2$, $I = [-2, 2]$, serie derivata come sopra.
\end{solution}

\newpage

\section*{Esercizio 2}

\textbf{Testo:} Determinare il polinomio di McLaurin di quarto grado di $f(x) = xe^{-x} + x \log(1+x)$. Quanto vale la derivata decima di f in 0?

\begin{solution}
Scomponiamo $f(x) = g(x) + h(x)$ dove $g(x) = xe^{-x}$ e $h(x) = x\log(1+x)$.

\textbf{Per $g(x) = xe^{-x}$:}

Sappiamo che $e^{-x} = 1 - x + \frac{x^2}{2} - \frac{x^3}{6} + \frac{x^4}{24} - \frac{x^5}{120} + O(x^6)$.

Quindi:
\[
xe^{-x} = x\left(1 - x + \frac{x^2}{2} - \frac{x^3}{6} + \frac{x^4}{24}\right) + O(x^6)
\]
\[
= x - x^2 + \frac{x^3}{2} - \frac{x^4}{6} + \frac{x^5}{24} + O(x^6)
\]

\textbf{Per $h(x) = x\log(1+x)$:}

Sappiamo che $\log(1+x) = x - \frac{x^2}{2} + \frac{x^3}{3} - \frac{x^4}{4} + \frac{x^5}{5} + O(x^6)$.

Quindi:
\[
x\log(1+x) = x\left(x - \frac{x^2}{2} + \frac{x^3}{3} - \frac{x^4}{4} + \frac{x^5}{5}\right) + O(x^7)
\]
\[
= x^2 - \frac{x^3}{2} + \frac{x^4}{3} - \frac{x^5}{4} + \frac{x^6}{5} + O(x^7)
\]

\textbf{Combinando:}
\[
f(x) = xe^{-x} + x\log(1+x)
\]
\[
= \left(x - x^2 + \frac{x^3}{2} - \frac{x^4}{6}\right) + \left(x^2 - \frac{x^3}{2} + \frac{x^4}{3}\right) + O(x^5)
\]
\[
= x + (-1 + 1)x^2 + \left(\frac{1}{2} - \frac{1}{2}\right)x^3 + \left(-\frac{1}{6} + \frac{1}{3}\right)x^4 + O(x^5)
\]
\[
= x + 0 \cdot x^2 + 0 \cdot x^3 + \frac{1}{6}x^4 + O(x^5)
\]

\textbf{Polinomio di McLaurin di quarto grado:}
\[
P_4(x) = x + \frac{x^4}{6}
\]

\textbf{Per la derivata decima:}

Il termine $x^{10}$ può venire solo da termini di ordine superiore negli sviluppi. 

Da $xe^{-x}$: il coefficiente di $x^{10}$ è $\frac{(-1)^9}{9!} = -\frac{1}{9!}$.

Da $x\log(1+x)$: il coefficiente di $x^{10}$ è $\frac{(-1)^8}{9} = \frac{1}{9}$.

Il coefficiente totale di $x^{10}$ è: $-\frac{1}{9!} + \frac{1}{9}$.

Quindi $f^{(10)}(0) = 10! \left(-\frac{1}{9!} + \frac{1}{9}\right) = -10 + \frac{10!}{9}$.

Calcolando: $\frac{10!}{9} = \frac{3628800}{9} = 403200$.

\textbf{Risposta:} $f^{(10)}(0) = 403200 - 10 = 403190$.
\end{solution}

\newpage

\section*{Esercizio 3}

\textbf{Testo:} Sia $f: \mathbb{R} \rightarrow \mathbb{R}$ la funzione periodica di periodo $2\pi$ data su $[-\pi, \pi)$ da
\[ \begin{cases} 3 & x \in [-\pi,0) \\ 1 & x \in [0,\pi) \end{cases} \]
\begin{enumerate}
    \item[a)] Determinare l'espressione di f e disegnare il suo grafico sull'intervallo $[-3\pi, 3\pi]$.
    \item[b)] Dire se la serie di Fourier converge puntualmente su $\mathbb{R}$ e determinarne il valore della somma su $[-3\pi, 3\pi]$.
\end{enumerate}

\begin{solution}
\textbf{Punto a)} La funzione f è periodica di periodo $2\pi$ e definita su $[-\pi, \pi)$ da:
\[
f(x) = \begin{cases} 3 & x \in [-\pi,0) \\ 1 & x \in [0,\pi) \end{cases}
\]

Per periodicità, f si estende a tutto $\mathbb{R}$ ripetendo questo pattern ogni $2\pi$.

Su $[-3\pi, 3\pi]$:
\begin{itemize}
    \item $[-3\pi, -2\pi)$: ripete il pattern, quindi $f(x) = 1$ su $[-3\pi, -2.5\pi)$ e $f(x) = 3$ su $[-2.5\pi, -2\pi)$
    \item $[-2\pi, -\pi)$: $f(x) = 1$ su $[-2\pi, -1.5\pi)$ e $f(x) = 3$ su $[-1.5\pi, -\pi)$
    \item $[-\pi, 0)$: $f(x) = 3$
    \item $[0, \pi)$: $f(x) = 1$
    \item $[\pi, 2\pi)$: $f(x) = 3$ su $[\pi, 1.5\pi)$ e $f(x) = 1$ su $[1.5\pi, 2\pi)$
    \item $[2\pi, 3\pi]$: $f(x) = 3$ su $[2\pi, 2.5\pi)$ e $f(x) = 1$ su $[2.5\pi, 3\pi]$
\end{itemize}

Il grafico è una funzione a gradini che alterna tra i valori 1 e 3.

\vspace{0.5cm}

\textbf{Punto b)} La funzione f ha discontinuità di salto nei punti $x = n\pi$ per $n \in \mathbb{Z}$.

Secondo il teorema di convergenza di Fourier:
\begin{itemize}
    \item Nei punti di continuità: la serie converge a $f(x)$
    \item Nei punti di discontinuità di salto: la serie converge alla media dei limiti destro e sinistro
\end{itemize}

Per i punti di discontinuità $x = n\pi$:
\begin{itemize}
    \item Se $n$ è pari: $\lim_{x \to n\pi^-} f(x) = 1$, $\lim_{x \to n\pi^+} f(x) = 3$, media = 2
    \item Se $n$ è dispari: $\lim_{x \to n\pi^-} f(x) = 3$, $\lim_{x \to n\pi^+} f(x) = 1$, media = 2
\end{itemize}

\textbf{Convergenza su $[-3\pi, 3\pi]$:}
\begin{itemize}
    \item Per $x \in (-3\pi, -2\pi) \cup (-2\pi, -\pi) \cup (-\pi, 0) \cup (0, \pi) \cup (\pi, 2\pi) \cup (2\pi, 3\pi)$: la serie converge a $f(x)$
    \item Per $x \in \{-3\pi, -2\pi, -\pi, 0, \pi, 2\pi, 3\pi\}$: la serie converge a 2
\end{itemize}

\textbf{Conclusione:} La serie di Fourier converge puntualmente su tutto $\mathbb{R}$.
\end{solution}

\newpage

\section*{Esercizio 4}

\textbf{Testo:} Sia $f(x,y) = e^{2x^{2}+2y^{2}-4y+2}$.
\begin{enumerate}
    \item[a)] Stabilire se f è differenziabile sul suo dominio e determinarne la derivata direzionale in $P=(0,3)$ lungo $v=(1,1)$.
    \item[b)] Dire se l'insieme $D = \{(x,y) \in \mathbb{R}^{2} | x^{2}+y^{2} \le 4\}$ è aperto o chiuso, limitato e connesso, e determinarne l'interno e la frontiera C.
    \item[c)] Determinare, se esistono, i punti di massimo e minimo relativo di f sull'interno di D.
    \item[d)] Determinare, se esistono, punti di massimo e minimo assoluto di f su C.
\end{enumerate}

\begin{solution}
\textbf{Punto a)} La funzione $f(x,y) = e^{2x^{2}+2y^{2}-4y+2}$ ha dominio $\mathbb{R}^2$.

Calcoliamo le derivate parziali:
\[
\frac{\partial f}{\partial x} = e^{2x^{2}+2y^{2}-4y+2} \cdot 4x = 4xe^{2x^{2}+2y^{2}-4y+2}
\]
\[
\frac{\partial f}{\partial y} = e^{2x^{2}+2y^{2}-4y+2} \cdot (4y - 4) = 4(y-1)e^{2x^{2}+2y^{2}-4y+2}
\]

Entrambe sono continue su $\mathbb{R}^2$, quindi f è differenziabile.

Il gradiente in $P = (0,3)$ è:
\[
\nabla f(0,3) = (4 \cdot 0 \cdot e^{2 \cdot 0 + 2 \cdot 9 - 4 \cdot 3 + 2}, 4(3-1)e^{2 \cdot 0 + 2 \cdot 9 - 4 \cdot 3 + 2})
\]
\[
= (0, 8e^{18 - 12 + 2}) = (0, 8e^8)
\]

Il vettore unitario nella direzione di $v = (1,1)$ è $\hat{v} = \frac{(1,1)}{\sqrt{2}}$.

La derivata direzionale è:
\[
\frac{\partial f}{\partial v}(0,3) = \nabla f(0,3) \cdot \hat{v} = (0, 8e^8) \cdot \frac{(1,1)}{\sqrt{2}} = \frac{8e^8}{\sqrt{2}} = 4\sqrt{2}e^8
\]

\vspace{0.5cm}

\textbf{Punto b)} L'insieme $D = \{(x,y) \in \mathbb{R}^{2} | x^{2}+y^{2} \le 4\}$ è:
\begin{itemize}
    \item Chiuso (contiene la frontiera)
    \item Non aperto (contiene punti di frontiera)
    \item Limitato (contenuto nel cerchio di raggio 2)
    \item Connesso (ogni due punti possono essere collegati da un cammino)
\end{itemize}

\textbf{Interno:} $\text{int}(D) = \{(x,y) \in \mathbb{R}^{2} | x^{2}+y^{2} < 4\}$

\textbf{Frontiera:} $C = \partial D = \{(x,y) \in \mathbb{R}^{2} | x^{2}+y^{2} = 4\}$

\vspace{0.5cm}

\textbf{Punto c)} Sull'interno di D, cerchiamo i punti critici di f:
\[
\begin{cases}
\frac{\partial f}{\partial x} = 4xe^{2x^{2}+2y^{2}-4y+2} = 0 \\
\frac{\partial f}{\partial y} = 4(y-1)e^{2x^{2}+2y^{2}-4y+2} = 0
\end{cases}
\]

Dalla prima equazione: $x = 0$ (poiché $e^{2x^{2}+2y^{2}-4y+2} > 0$).
Dalla seconda equazione: $y = 1$.

Il punto critico è $(0, 1)$. Verifichiamo che sia nell'interno: $0^2 + 1^2 = 1 < 4$ (vero).

Per classificarlo, calcoliamo la matrice Hessiana:
\[
H_f = \begin{pmatrix}
\frac{\partial^2 f}{\partial x^2} & \frac{\partial^2 f}{\partial x \partial y} \\
\frac{\partial^2 f}{\partial y \partial x} & \frac{\partial^2 f}{\partial y^2}
\end{pmatrix}
\]

In $(0,1)$: $f(0,1) = e^{0 + 2 - 4 + 2} = e^0 = 1$.

Calcolando: $H_f(0,1) = \begin{pmatrix} 4e^0 & 0 \\ 0 & 4e^0 \end{pmatrix} = \begin{pmatrix} 4 & 0 \\ 0 & 4 \end{pmatrix}$.

Gli autovalori sono entrambi 4 > 0, quindi $(0,1)$ è un minimo relativo.

\vspace{0.5cm}

\textbf{Punto d)} Sulla frontiera $C$, usiamo i moltiplicatori di Lagrange per ottimizzare $f(x,y) = e^{2x^{2}+2y^{2}-4y+2}$ sotto il vincolo $g(x,y) = x^2 + y^2 - 4 = 0$.

Il sistema è:
\[
\begin{cases}
4xe^{2x^{2}+2y^{2}-4y+2} = 2\lambda x \\
4(y-1)e^{2x^{2}+2y^{2}-4y+2} = 2\lambda y \\
x^2 + y^2 = 4
\end{cases}
\]

Se $x \neq 0$: $2e^{2x^{2}+2y^{2}-4y+2} = \lambda$.
Se $y \neq 0$: $\frac{2(y-1)e^{2x^{2}+2y^{2}-4y+2}}{y} = \lambda$.

Uguagliando: $2e^{2x^{2}+2y^{2}-4y+2} = \frac{2(y-1)e^{2x^{2}+2y^{2}-4y+2}}{y}$.

Semplificando: $y = y - 1$, che è impossibile.

Consideriamo i casi $x = 0$ o $y = 0$:

\textbf{Caso $x = 0$:} $y^2 = 4$, quindi $y = \pm 2$.
\begin{itemize}
    \item $(0, 2)$: $f(0, 2) = e^{0 + 8 - 8 + 2} = e^2$
    \item $(0, -2)$: $f(0, -2) = e^{0 + 8 + 8 + 2} = e^{18}$
\end{itemize}

\textbf{Caso $y = 0$:} $x^2 = 4$, quindi $x = \pm 2$.
\begin{itemize}
    \item $(2, 0)$: $f(2, 0) = e^{8 + 0 + 0 + 2} = e^{10}$
    \item $(-2, 0)$: $f(-2, 0) = e^{8 + 0 + 0 + 2} = e^{10}$
\end{itemize}

\textbf{Conclusione:}
\begin{itemize}
    \item Massimo assoluto su C: $e^{18}$ nel punto $(0, -2)$
    \item Minimo assoluto su C: $e^2$ nel punto $(0, 2)$
\end{itemize}
\end{solution}

\end{document}
