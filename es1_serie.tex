\section{Esercizio 1: Serie Numeriche e di Potenze}
Questo è spesso il primo scoglio. L'obiettivo è capire il "carattere" di una serie. La chiave è la sistematicità.

\subsection{Studio della Convergenza di Serie Numeriche}
\begin{strategia}
\textbf{Flowchart mentale:}
\begin{enumerate}
    \item \textbf{Condizione Necessaria:} Il termine $a_n \to 0$? Se NO $\implies$ DIVERGE. Se SÌ, procedi. 
    \item \textbf{Segno:} La serie è a termini positivi? (o definitivamente positivi). Se SÌ, usa i criteri per serie positive (confronto asintotico è il più potente). Se NO, vai al punto 3. 
    \item \textbf{Convergenza Assoluta:} Studia $\sum |a_n|$ (che è a termini positivi). Se converge, hai finito: la serie converge ASSOLUTAMENTE (e quindi anche semplicemente). Se diverge, non puoi concludere nulla sulla convergenza semplice, vai al punto 4. 
    \item \textbf{Convergenza Semplice (Criterio di Leibniz):} La serie è a \textbf{segni alterni}, della forma $\sum (-1)^n b_n$ con $b_n>0$? Se SÌ, verifica le due condizioni di Leibniz ($b_n \to 0$ e $b_n$ decrescente). Se valgono, la serie converge SEMPLICEMENTE. 
\end{enumerate}
\end{strategia}

\subsubsection{Analisi Dettagliata dei Criteri}
\begin{enumerate}
    \item \textbf{Condizione Necessaria:} $\lim_{n \to \infty} a_n = 0$. 
    \begin{errore}
    Se il limite è 0, \textbf{non puoi concludere nulla}! La condizione è solo necessaria, non sufficiente. Dire "siccome $a_n \to 0$ la serie converge" è un errore grave che invalida l'esercizio. 
    \end{errore}

    \begin{info}
    \textbf{Come Scegliere la Serie di Confronto Giusta:}
    \begin{itemize}
        \item \textbf{Se $a_n$ contiene potenze di $n$:} usa la serie armonica generalizzata $\sum \frac{1}{n^\alpha}$
        \item \textbf{Se $a_n$ contiene esponenziali:} usa la serie geometrica $\sum q^n$  
        \item \textbf{Se $a_n$ contiene logaritmi:} spesso confronta con $\sum \frac{1}{n (\ln n)^\beta}$
        \item \textbf{Se $a_n$ contiene fattoriali:} usa Stirling e confronta con serie esponenziali
        \item \textbf{Regola d'oro:} semplifica $a_n$ fino al termine dominante per $n \to \infty$
    \end{itemize}
    \end{info}

    \begin{strategia}
    \textbf{Algoritmo Sistematico per il Confronto Asintotico:}
    \begin{enumerate}
        \item \textbf{Semplifica $a_n$:} Usa sviluppi di Taylor/MacLaurin e equivalenze asintotiche
        \item \textbf{Identifica il termine dominante:} Quello che cresce/decresce più velocemente
        \item \textbf{Scegli la serie di confronto $\sum b_n$:} Basata sul termine dominante
        \item \textbf{Calcola $L = \lim_{n \to \infty} \frac{a_n}{b_n}$}
        \item \textbf{Applica le regole del confronto asintotico}
        \item \textbf{Se $L = 0$ o $L = +\infty$:} Potrebbe servire una serie di confronto diversa
    \end{enumerate}
    \end{strategia}

    \begin{esempio}
    \textbf{Caso Delicato - Quando il Primo Confronto Fallisce:}
    
    Studiare $\sum_{n=1}^{+\infty} \frac{\ln n}{n^{3/2}}$
    
    \textbf{Primo tentativo:} Confronto con $\sum \frac{1}{n}$ (serie armonica)
    
    $L = \lim_{n \to \infty} \frac{\ln n / n^{3/2}}{1/n} = \lim_{n \to \infty} \frac{\ln n}{n^{1/2}} = 0$
    
    Poiché $\sum \frac{1}{n}$ diverge e $L = 0$, NON possiamo concludere.
    
    \textbf{Secondo tentativo:} Confronto con $\sum \frac{1}{n^{3/2}}$ (converge perché $3/2 > 1$)
    
    $L = \lim_{n \to \infty} \frac{\ln n / n^{3/2}}{1/n^{3/2}} = \lim_{n \to \infty} \ln n = +\infty$
    
    Poiché $\sum \frac{1}{n^{3/2}}$ converge e $L = +\infty$, NON possiamo concludere.
    
    \textbf{Confronto giusto:} Serie $\sum \frac{1}{n^{3/2}} \cdot \frac{\ln n}{C}$ per $C$ costante appropriata
    
    Meglio usare il \textbf{criterio integrale}: $\int_2^{+\infty} \frac{\ln x}{x^{3/2}} dx$ converge $\Rightarrow$ la serie CONVERGE.
    \end{esempio}

    \item \textbf{Convergenza Assoluta (Studio di $\sum |a_n|$):} 
    \begin{itemize}
        \item \textbf{Confronto Asintotico (il più potente):} Semplifica $|a_n|$ per $n \to \infty$ usando gli sviluppi di MacLaurin o le equivalenze asintotiche notevoli (es. $\sin(x) \sim x$, $\ln(1+x) \sim x$, $e^x-1 \sim x$ per $x \to 0$). Confrontala con la serie armonica generalizzata $\sum \frac{1}{n^\alpha}$ (converge se $\alpha > 1$) o la serie geometrica $\sum q^n$ (converge se $|q| < 1$). 
        \item \textbf{Criterio del Rapporto:} Utile con fattoriali ($n!$) o termini esponenziali ($k^n$). Calcola $L = \lim_{n \to \infty} \frac{|a_{n+1}|}{|a_n|}$. Se $L<1$ converge assolutamente, se $L>1$ diverge, se $L=1$ è inconclusivo. 
        \item \textbf{Criterio della Radice:} Utile con potenze $n$-esime. Calcola $L = \lim_{n \to \infty} \sqrt[n]{|a_n|}$. Le conclusioni sono le stesse del criterio del rapporto. 
    \end{itemize}
    
    \begin{info}
    \textbf{Tabella Riassuntiva - Confronto Asintotico:}
    
    \begin{center}
    \begin{tabular}{|c|c|c|c|}
    \hline
    \textbf{Valore di $L$} & \textbf{Se $\sum b_n$ converge} & \textbf{Se $\sum b_n$ diverge} & \textbf{Interpretazione} \\
    \hline
    $L = 0$ & $\sum a_n$ converge & Inconclusivo & $a_n \ll b_n$ \\
    \hline
    $0 < L < +\infty$ & $\sum a_n$ converge & $\sum a_n$ diverge & $a_n \sim L \cdot b_n$ \\
    \hline
    $L = +\infty$ & Inconclusivo & $\sum a_n$ diverge & $a_n \gg b_n$ \\
    \hline
    \end{tabular}
    \end{center}
    
    \textbf{Serie di Confronto Standard:}
    \begin{itemize}
        \item $\sum \frac{1}{n^\alpha}$: converge se $\alpha > 1$, diverge se $\alpha \leq 1$
        \item $\sum q^n$: converge se $|q| < 1$, diverge se $|q| \geq 1$  
        \item $\sum \frac{1}{n (\ln n)^\beta}$: converge se $\beta > 1$, diverge se $\beta \leq 1$ (per $n \geq 2$)
        \item $\sum \frac{1}{n (\ln n)(\ln \ln n)^\gamma}$: converge se $\gamma > 1$ (per $n \geq 3$)
    \end{itemize}
    \end{info}

    \begin{info}
    \textbf{Equivalenze Asintotiche Notevoli (per $x \to 0$):}
    
    \begin{center}
    \begin{tabular}{|c|c|c|}
    \hline
    \textbf{Funzione} & \textbf{Equivalenza asintotica} & \textbf{Dominio} \\
    \hline
    $\sin x$ & $x$ & $x \to 0$ \\
    \hline
    $\tan x$ & $x$ & $x \to 0$ \\
    \hline
    $\arcsin x$ & $x$ & $x \to 0$ \\
    \hline
    $\arctan x$ & $x$ & $x \to 0$ \\
    \hline
    $\ln(1+x)$ & $x$ & $x \to 0$ \\
    \hline
    $e^x - 1$ & $x$ & $x \to 0$ \\
    \hline
    $(1+x)^\alpha - 1$ & $\alpha x$ & $x \to 0, \alpha \neq 0$ \\
    \hline
    $\sqrt{1+x} - 1$ & $\frac{x}{2}$ & $x \to 0$ \\
    \hline
    $\cos x - 1$ & $-\frac{x^2}{2}$ & $x \to 0$ \\
    \hline
    $1 - \cos x$ & $\frac{x^2}{2}$ & $x \to 0$ \\
    \hline
    $\cosh x - 1$ & $\frac{x^2}{2}$ & $x \to 0$ \\
    \hline
    $\sinh x$ & $x$ & $x \to 0$ \\
    \hline
    $\tanh x$ & $x$ & $x \to 0$ \\
    \hline
    \end{tabular}
    \end{center}

    \textbf{Equivalenze per $n \to +\infty$:}
    \begin{center}
    \begin{tabular}{|c|c|}
    \hline
    \textbf{Funzione} & \textbf{Equivalenza asintotica} \\
    \hline
    $\sin\left(\frac{1}{n}\right)$ & $\frac{1}{n}$ \\
    \hline
    $\ln\left(1 + \frac{1}{n}\right)$ & $\frac{1}{n}$ \\
    \hline
    $e^{1/n} - 1$ & $\frac{1}{n}$ \\
    \hline
    $\left(1 + \frac{1}{n}\right)^n - e$ & $-\frac{e}{2n}$ \\
    \hline
    $\sqrt{n+1} - \sqrt{n}$ & $\frac{1}{2\sqrt{n}}$ \\
    \hline
    $\ln(n+1) - \ln n$ & $\frac{1}{n}$ \\
    \hline
    \end{tabular}
    \end{center}

    \textbf{Formule di Stirling:}
    \begin{itemize}
        \item $n! \sim \sqrt{2\pi n}\left(\frac{n}{e}\right)^n$ per $n \to +\infty$
        \item $\binom{2n}{n} \sim \frac{4^n}{\sqrt{\pi n}}$ per $n \to +\infty$
        \item $\frac{(n!)^2}{(2n)!} \sim \frac{1}{\sqrt{\pi n}}$ per $n \to +\infty$
    \end{itemize}
    \end{info}

    \begin{esempio}
    \textbf{Applicazione delle Equivalenze Asintotiche:}
    
    \textbf{1.} $\sum_{n=1}^{+\infty} \frac{\sin(1/n^2)}{n}$
    
    Usando $\sin(1/n^2) \sim 1/n^2$ per $n \to +\infty$:
    \[ a_n = \frac{\sin(1/n^2)}{n} \sim \frac{1/n^2}{n} = \frac{1}{n^3} \]
    Confronto con $\sum \frac{1}{n^3}$ (converge perché $3 > 1$) $\Rightarrow$ la serie CONVERGE.
    
    \textbf{2.} $\sum_{n=1}^{+\infty} \frac{\ln(1 + 1/\sqrt{n})}{n}$
    
    Usando $\ln(1 + 1/\sqrt{n}) \sim 1/\sqrt{n}$ per $n \to +\infty$:
    \[ a_n = \frac{\ln(1 + 1/\sqrt{n})}{n} \sim \frac{1/\sqrt{n}}{n} = \frac{1}{n^{3/2}} \]
    Confronto con $\sum \frac{1}{n^{3/2}}$ (converge perché $3/2 > 1$) $\Rightarrow$ la serie CONVERGE.
    
    \textbf{3.} $\sum_{n=1}^{+\infty} \frac{e^{1/n} - 1}{\sqrt{n}}$
    
    Usando $e^{1/n} - 1 \sim 1/n$ per $n \to +\infty$:
    \[ a_n = \frac{e^{1/n} - 1}{\sqrt{n}} \sim \frac{1/n}{\sqrt{n}} = \frac{1}{n^{3/2}} \]
    Confronto con $\sum \frac{1}{n^{3/2}}$ (converge perché $3/2 > 1$) $\Rightarrow$ la serie CONVERGE.
    
    \textbf{4.} $\sum_{n=1}^{+\infty} \frac{(1+1/n)^n - e}{n^2}$
    
    Usando $(1+1/n)^n - e \sim -\frac{e}{2n}$ per $n \to +\infty$:
    \[ a_n = \frac{(1+1/n)^n - e}{n^2} \sim \frac{-e/(2n)}{n^2} = \frac{-e}{2n^3} \]
    Confronto con $\sum \frac{1}{n^3}$ (converge perché $3 > 1$) $\Rightarrow$ la serie CONVERGE ASSOLUTAMENTE.
    \end{esempio}
    
    \begin{strategia}
    \textbf{Confronto Asintotico - Casistica Completa:}
    
    Sia $\sum a_n$ la serie da studiare e $\sum b_n$ una serie di confronto (es. armonica generalizzata $\sum \frac{1}{n^\alpha}$ o geometrica $\sum q^n$). Calcoliamo:
    \[ L = \lim_{n \to \infty} \frac{a_n}{b_n} \]
    
    \textbf{Risultati e Conclusioni:}
    \begin{itemize}
        \item \textbf{Se $L = 0$:} $a_n$ va a zero più velocemente di $b_n$
        \begin{itemize}
            \item Se $\sum b_n$ CONVERGE $\Rightarrow$ $\sum a_n$ CONVERGE
            \item Se $\sum b_n$ DIVERGE $\Rightarrow$ NESSUNA conclusione
        \end{itemize}
        
        \item \textbf{Se $L = c > 0$ (finito):} $a_n$ e $b_n$ hanno lo stesso comportamento asintotico
        \begin{itemize}
            \item $\sum a_n$ e $\sum b_n$ hanno lo \textbf{stesso carattere}
            \item Se $\sum b_n$ CONVERGE $\Rightarrow$ $\sum a_n$ CONVERGE
            \item Se $\sum b_n$ DIVERGE $\Rightarrow$ $\sum a_n$ DIVERGE
        \end{itemize}
        
        \item \textbf{Se $L = +\infty$:} $a_n$ va a zero più lentamente di $b_n$
        \begin{itemize}
            \item Se $\sum b_n$ DIVERGE $\Rightarrow$ $\sum a_n$ DIVERGE
            \item Se $\sum b_n$ CONVERGE $\Rightarrow$ NESSUNA conclusione
        \end{itemize}
    \end{itemize}
    \end{strategia}

    \begin{esempio}
    \textbf{Caso $L = 0$:} $\sum_{n=1}^{+\infty} \frac{1}{n^2 \ln n}$ vs $\sum \frac{1}{n^2}$
    
    $L = \lim_{n \to \infty} \frac{1/(n^2 \ln n)}{1/n^2} = \lim_{n \to \infty} \frac{1}{\ln n} = 0$
    
    Poiché $\sum \frac{1}{n^2}$ converge (serie armonica generalizzata con $\alpha = 2 > 1$) e $L = 0$, allora $\sum \frac{1}{n^2 \ln n}$ CONVERGE.
    \end{esempio}

    \begin{esempio}
    \textbf{Caso $L = c > 0$:} $\sum_{n=1}^{+\infty} \frac{2n + 5}{3n^2 + n}$ vs $\sum \frac{1}{n}$
    
    $L = \lim_{n \to \infty} \frac{(2n + 5)/(3n^2 + n)}{1/n} = \lim_{n \to \infty} \frac{2n + 5}{3n + 1} = \frac{2}{3}$
    
    Poiché $\sum \frac{1}{n}$ diverge (serie armonica) e $L = \frac{2}{3} > 0$, allora $\sum \frac{2n + 5}{3n^2 + n}$ DIVERGE.
    \end{esempio}

    \begin{esempio}
    \textbf{Caso $L = +\infty$:} $\sum_{n=1}^{+\infty} \frac{1}{\sqrt{n}}$ vs $\sum \frac{1}{n^2}$
    
    $L = \lim_{n \to \infty} \frac{1/\sqrt{n}}{1/n^2} = \lim_{n \to \infty} \frac{n^2}{\sqrt{n}} = \lim_{n \to \infty} n^{3/2} = +\infty$
    
    Poiché $\sum \frac{1}{n^2}$ converge ma $L = +\infty$, NON possiamo concludere nulla con questo confronto. Dobbiamo cambiare serie di confronto: confrontiamo con $\sum \frac{1}{\sqrt{n}} = \sum \frac{1}{n^{1/2}}$ (serie armonica generalizzata con $\alpha = 1/2 < 1$), che DIVERGE.
    \end{esempio}

    \begin{esempio}
    Studiare la convergenza di $\sum_{n=1}^{+\infty} 2^n \sin\left(\frac{1}{5^n}\right)$.
    La serie è a termini positivi. Per $n \to \infty$, l'argomento del seno $1/5^n \to 0$. Usiamo il confronto asintotico: $\sin(1/5^n) \sim 1/5^n$. 
    Quindi, la nostra serie ha lo stesso carattere di:
    \[ \sum_{n=1}^{+\infty} 2^n \cdot \frac{1}{5^n} = \sum_{n=1}^{+\infty} \left(\frac{2}{5}\right)^n \]
    Questa è una serie geometrica di ragione $q=2/5$.  Poiché $|q|<1$, la serie converge. 
    \end{esempio}

    \item \textbf{Convergenza Semplice (se non c'è conv. assoluta):}
    \begin{itemize}
        \item \textbf{Criterio di Leibniz:} Per una serie a segno alterno $\sum (-1)^n b_n$ con $b_n \ge 0$, verifica \textbf{entrambe} le condizioni: 
        \begin{enumerate}
            \item $\lim_{n \to \infty} b_n = 0$ (la condizione necessaria!). 
            \item $b_n$ è \textbf{definitivamente decrescente}. Per provarlo, puoi studiare il segno della derivata della funzione associata $f(x)$ (se $f'(x) < 0$) o verificare che $b_{n+1} \le b_n$. 
        \end{enumerate}

    \begin{strategia}
    \textbf{Metodi Rapidi per Verificare la Monotonia Decrescente:}
    
    Data una successione $b_n = f(n)$, per verificare che sia definitivamente decrescente:
    
    \textbf{Metodo 1 - Studio della Derivata (IL PIÙ VELOCE):}
    \begin{enumerate}
        \item Considera la funzione $f(x)$ dove $x$ è una variabile continua
        \item Calcola $f'(x)$
        \item Se $f'(x) < 0$ per $x \geq N$ (per qualche $N$ sufficientemente grande), allora $b_n$ è definitivamente decrescente
    \end{enumerate}
    
    \textbf{Metodo 2 - Rapporto Consecutivo:}
    \begin{enumerate}
        \item Calcola $\frac{b_{n+1}}{b_n}$
        \item Se $\frac{b_{n+1}}{b_n} < 1$ per $n \geq N$, allora $b_n$ è definitivamente decrescente
        \item Questo metodo è utile quando $b_n$ contiene fattoriali o potenze
    \end{enumerate}
    
    \textbf{Metodo 3 - Differenza Consecutiva:}
    \begin{enumerate}
        \item Calcola $b_{n+1} - b_n$
        \item Se $b_{n+1} - b_n < 0$ per $n \geq N$, allora $b_n$ è definitivamente decrescente
    \end{enumerate}
    \end{strategia}

    \begin{esempio}
    \textbf{Esempi Pratici di Verifica della Monotonia:}
    
    \textbf{1.} $b_n = \frac{1}{\sqrt{n} + \ln n}$
    
    \textbf{Metodo 1:} $f(x) = \frac{1}{\sqrt{x} + \ln x}$
    \[ f'(x) = -\frac{\frac{1}{2\sqrt{x}} + \frac{1}{x}}{(\sqrt{x} + \ln x)^2} = -\frac{\frac{x + 2\sqrt{x}}{2x\sqrt{x}}}{(\sqrt{x} + \ln x)^2} < 0 \]
    per $x > 0$. Quindi $b_n$ è decrescente $\checkmark$
    
    \textbf{2.} $b_n = \frac{n!}{n^n}$
    
    \textbf{Metodo 2:} $\frac{b_{n+1}}{b_n} = \frac{(n+1)!/(n+1)^{n+1}}{n!/n^n} = \frac{(n+1) \cdot n!}{(n+1)^{n+1}} \cdot \frac{n^n}{n!} = \frac{n^n}{(n+1)^n} = \left(\frac{n}{n+1}\right)^n$
    
    Poiché $\frac{n}{n+1} < 1$, abbiamo $\left(\frac{n}{n+1}\right)^n < 1$. Quindi $b_n$ è decrescente $\checkmark$
    
    \textbf{3.} $b_n = \frac{\ln n}{n}$
    
    \textbf{Metodo 1:} $f(x) = \frac{\ln x}{x}$
    \[ f'(x) = \frac{\frac{1}{x} \cdot x - \ln x \cdot 1}{x^2} = \frac{1 - \ln x}{x^2} \]
    
    $f'(x) < 0$ quando $1 - \ln x < 0$, cioè quando $\ln x > 1$, ossia $x > e$.
    Quindi $b_n$ è decrescente per $n \geq 3$ $\checkmark$
    
    \textbf{4.} $b_n = \frac{1}{n^2 + n}$ 
    
    \textbf{Metodo 3:} $b_{n+1} - b_n = \frac{1}{(n+1)^2 + (n+1)} - \frac{1}{n^2 + n}$
    \[ = \frac{1}{(n+1)(n+2)} - \frac{1}{n(n+1)} = \frac{n - (n+2)}{n(n+1)(n+2)} = \frac{-2}{n(n+1)(n+2)} < 0 \]
    
    Quindi $b_n$ è decrescente $\checkmark$
    \end{esempio}

    \begin{errore}
    \textbf{Errori Comuni nella Verifica della Monotonia:}
    \begin{itemize}
        \item $\times$ \textit{Sbagliato:} Verificare solo che $b_n \to 0$ e concludere che la serie converge per Leibniz
        \item $\times$ \textit{Sbagliato:} Calcolare $f'(0)$ invece di studiare il segno di $f'(x)$ per $x$ grande
        \item $\times$ \textit{Sbagliato:} Non verificare che la monotonia vale "definitivamente" (cioè da un certo $n$ in poi)
        \item $\checkmark$ \textit{Giusto:} Verificare ENTRAMBE le condizioni: $b_n \to 0$ E $b_n$ definitivamente decrescente
    \end{itemize}
    \end{errore}
        \item \textbf{Stima dell'errore (Leibniz):} Se una serie a termini alterni converge a $S$, l'errore commesso troncando la serie alla somma parziale $S_N$ è minore in valore assoluto del primo termine trascurato: $|S - S_N| \le b_{N+1}$. 
    \end{itemize}
\end{enumerate}

\subsubsection{Tecniche Avanzate per Serie Speciali (dalle Esercitazioni)}

\begin{info}
\textbf{Serie con Fattoriali e Coefficienti Binomiali:}
\begin{itemize}
    \item Per serie del tipo $\sum \frac{(n!)^2}{(2n)!}$, usa la formula di Stirling: $n! \sim \sqrt{2\pi n}\left(\frac{n}{e}\right)^n$ per $n \to \infty$.
    \item Per coefficienti binomiali $\binom{an}{bn}$, ricorda che $\binom{2n}{n} \sim \frac{4^n}{\sqrt{\pi n}}$.
    \item Per serie con $\frac{n^n}{e^n}$, usa il fatto che $\frac{n^n}{e^n} = \left(\frac{n}{e}\right)^n$ e confronta con una serie geometrica.
\end{itemize}
\end{info}

\begin{esempio}
Serie dalle esercitazioni: $\sum_{n=1}^{+\infty}\frac{(2n)!}{(3n)!}\frac{n^n}{e^n}$

Analizziamo separatamente i due fattori:
\begin{itemize}
    \item $\frac{(2n)!}{(3n)!} = \frac{1}{(2n+1)(2n+2)\cdots(3n)} \sim \frac{1}{n^n}$ (per confronto asintotico)
    \item $\frac{n^n}{e^n} = \left(\frac{n}{e}\right)^n$
\end{itemize}
Quindi $a_n \sim \frac{1}{n^n} \cdot \left(\frac{n}{e}\right)^n = \frac{1}{e^n}$, e la serie converge come una serie geometrica.
\end{esempio}

\begin{info}
\textbf{Serie con Logaritmi nelle Esercitazioni:}
Per serie del tipo $\sum (-1)^n \ln\left(1+\frac{1}{\sqrt{n}}\right)$:
\begin{itemize}
    \item Usa l'equivalenza $\ln(1+x) \sim x$ per $x \to 0$
    \item Quindi $\ln\left(1+\frac{1}{\sqrt{n}}\right) \sim \frac{1}{\sqrt{n}}$
    \item La serie diventa $\sum \frac{(-1)^n}{\sqrt{n}}$, che converge per Leibniz
\end{itemize}
\end{info}

\subsection{Serie di Potenze $\sum c_n (x-x_0)^n$}
\begin{enumerate}
    \item \textbf{Raggio di Convergenza $\rho$:} Si calcola \textbf{sempre} con il criterio del rapporto o della radice applicato al valore assoluto dei coefficienti, $|c_n|$. 
    \[ L = \lim_{n \to \infty} \left|\frac{c_{n+1}}{c_n}\right| \quad \text{oppure} \quad L = \lim_{n \to \infty} \sqrt[n]{|c_n|} \]
    Il raggio di convergenza è $\rho = \frac{1}{L}$. (Se $L=0 \implies \rho=+\infty$. Se $L=+\infty \implies \rho=0$). 
    \item \textbf{Intervallo di Convergenza:} La serie converge assolutamente (e quindi semplicemente) nell'intervallo aperto $(x_0 - \rho, x_0 + \rho)$. 
    \item \textbf{Studio agli Estremi:} Sostituisci $x = x_0 - \rho$ e $x = x_0 + \rho$ nell'espressione della serie. Ottieni due serie \textbf{numeriche} da studiare con i metodi del punto precedente. 
    \begin{errore}
    Dimenticarsi di studiare il comportamento agli estremi è uno degli errori più frequenti e costa punti preziosi. L'insieme di convergenza non è completo senza questa analisi. 
    \end{errore}
    \item \textbf{Insieme di Convergenza:} È l'unione dell'intervallo aperto e degli eventuali estremi in cui la serie converge. Può essere $(a,b)$, $[a,b)$, $(a,b]$ o $[a,b]$. 
    \item \textbf{Serie Derivata:} Se $\sum_{n=0}^{\infty} c_n x^n$ ha raggio di convergenza $\rho$, allora la serie derivata $\sum_{n=1}^{\infty} n c_n x^{n-1}$ ha lo stesso raggio di convergenza $\rho$. La funzione somma della serie derivata è la derivata della funzione somma originale nell'intervallo di convergenza.
    \item \textbf{Funzione Somma:} Se riconosci una serie di potenze come sviluppo di una funzione nota (es. geometrica, esponenziale), puoi calcolare direttamente la somma. Esempi:
    \begin{itemize}
        \item $\sum_{n=0}^{\infty} x^n = \frac{1}{1-x}$ per $|x| < 1$
        \item $\sum_{n=0}^{\infty} \frac{x^n}{n!} = e^x$ per ogni $x \in \mathbb{R}$
    \end{itemize}
\end{enumerate}

\subsubsection{Tecniche Speciali per Serie dalle Esercitazioni}
\begin{itemize}
    \item \textbf{Serie con Coefficienti Complessi:} Per serie del tipo $\sum \frac{2^n + 3^{-n}}{n^2} x^n$, il raggio è determinato dal termine dominante: $\rho = \frac{1}{2}$.
    \item \textbf{Ridotte e Stima dell'Errore:} Per una serie convergente $\sum a_n = S$, la ridotta $s_N = \sum_{n=1}^N a_n$ approssima $S$. Per serie a termini alterni che soddisfano Leibniz, l'errore è $|S - s_N| \leq |a_{N+1}|$.
    \item \textbf{Convergenza Totale:} Una serie di funzioni $\sum f_n(x)$ converge totalmente su un insieme $A$ se $\sum \|f_n\|_A < \infty$ dove $\|f_n\|_A = \sup_{x \in A} |f_n(x)|$.
\end{itemize}

\begin{esempio}
Serie dalle esercitazioni: $\sum_{n=1}^{+\infty} \frac{n+1}{n+2} \frac{x^{3n}}{3^n}$

Per trovare il raggio:
\begin{enumerate}
    \item Riscriviamo: $\sum_{n=1}^{+\infty} \frac{n+1}{n+2} \left(\frac{x^3}{3}\right)^n$
    \item Il coefficiente è $c_n = \frac{n+1}{n+2} \to 1$ per $n \to \infty$
    \item Applichiamo il criterio del rapporto alla serie in $y = x^3$: $\rho_y = 3$
    \item Quindi $|x^3| < 3 \Rightarrow |x| < \sqrt[3]{3}$
    \item Il raggio di convergenza è $\rho = \sqrt[3]{3}$
\end{enumerate}
\end{esempio}

\subsubsection{Esercizi Tipo Esame - Risolti Step by Step}

\begin{esempio}
\textbf{Esame 1 Luglio 2021 - Es. 1a:} $\displaystyle\sum_{n=1}^{+\infty} 2^{n} \sin\frac{1}{5^{n}}$

\textbf{Soluzione strategica:}
\begin{enumerate}
    \item \textbf{Osservazione:} Serie a termini positivi (per $n \ge 1$)
    \item \textbf{Confronto asintotico:} Per $n \to \infty$, $\frac{1}{5^n} \to 0$, quindi $\sin\frac{1}{5^n} \sim \frac{1}{5^n}$
    \item \textbf{Serie equivalente:} $\sum 2^n \cdot \frac{1}{5^n} = \sum \left(\frac{2}{5}\right)^n$
    \item \textbf{Conclusione:} Serie geometrica con $|q| = \frac{2}{5} < 1 \Rightarrow$ CONVERGE
\end{enumerate}
\end{esempio}

\begin{esempio}
\textbf{Esame 1 Luglio 2021 - Es. 1b:} $\displaystyle\sum_{n=1}^{+\infty} (-1)^{n} \frac{3}{\sqrt{n} + \log n}$

\textbf{Soluzione strategica:}
\begin{enumerate}
    \item \textbf{Convergenza assoluta:} Studio $\sum \frac{3}{\sqrt{n} + \log n}$
    \item \textbf{Comportamento asintotico:} $\sqrt{n} + \log n \sim \sqrt{n}$ per $n \to \infty$
    \item \textbf{Confronto:} $\frac{3}{\sqrt{n} + \log n} \sim \frac{3}{\sqrt{n}} = \frac{3}{n^{1/2}}$
    \item \textbf{Serie armonica generalizzata:} $\alpha = 1/2 < 1 \Rightarrow$ DIVERGE (no conv. assoluta)
    \item \textbf{Leibniz:} Serie alternante con $b_n = \frac{3}{\sqrt{n} + \log n}$
        \begin{itemize}
            \item $b_n \to 0$ $\checkmark$
            \item \(b_n\) decrescente? \(f(x) = \frac{3}{\sqrt{x} + \log x}\), \(f'(x) < 0\) per \(x > 1\) $\checkmark$
        \end{itemize}
    \item \textbf{Conclusione:} CONVERGE SEMPLICEMENTE (ma non assolutamente)
\end{enumerate}
\end{esempio}

\subsection{$\triangleright$ Errori Comuni e Trabocchetti dell'Esame}

\begin{errore}
\textbf{Serie di Potenze -- Confusione sul Raggio di Convergenza:}
\begin{itemize}
    \item $\times$ \textit{Sbagliato:} Dire che $\sum_{n=1}^{\infty} \frac{(2x)^n}{n}$ ha raggio $R=1$
    \item $\checkmark$ \textit{Giusto:} Prima riscrivere come $\sum_{n=1}^{\infty} \frac{2^n x^n}{n}$, poi $R = \frac{1}{2}$
    \item \textbf{Regola:} Il raggio si calcola sui coefficienti di $x^n$, non sul termine completo!
\end{itemize}
\end{errore}

\begin{errore}
\textbf{Criterio di Leibniz -- Dimenticare la Monotonia:}
\begin{itemize}
    \item $\times$ \textit{Sbagliato:} Verificare solo $b_n \to 0$ e concludere che converge
    \item $\checkmark$ \textit{Giusto:} Verificare ENTRAMBE: $b_n \to 0$ E $b_n$ decrescente
    \item \textbf{Trucco:} Per la monotonia, studia $f'(x) < 0$ della funzione associata
\end{itemize}
\end{errore}

\subsubsection{Tecniche per Serie ``Difficili'' degli Esami}

\begin{info}
\textbf{Serie con Prodotti Infiniti e Fattoriali Complessi:}
Per serie del tipo \(\sum \frac{(n!)^a}{(kn)!} \cdot g(n)\):
\begin{itemize}
    \item Usa la formula di Stirling: \(n! \sim \sqrt{2\pi n}\left(\frac{n}{e}\right)^n\)
    \item Per \(\frac{(n!)^2}{(2n)!}\): usa il fatto che \(\frac{(n!)^2}{(2n)!} \sim \frac{1}{\sqrt{\pi n}}\)
    \item Per \(\frac{(2n)!}{(3n)!}\): comportamento asintotico \(\sim \frac{1}{n^n}\)
\end{itemize}
\end{info}

\begin{esempio}
\textbf{Serie ``Impossibile'':} \(\displaystyle\sum_{n=1}^{\infty} \frac{1}{\binom{4n}{3n}}\)

\textbf{Trucco chiave:} \(\binom{4n}{3n} = \frac{(4n)!}{(3n)! \cdot n!}\)

Usando Stirling per grandi \(n\):
\begin{align}
\binom{4n}{3n} &\sim \frac{(4n)!}{(3n)! \cdot n!} \\
&\sim \frac{\sqrt{8\pi n}{(4n/e)}^{4n}}{\sqrt{6\pi n}{(3n/e)}^{3n} \cdot \sqrt{2\pi n}{(n/e)}^n} \\
&\sim \frac{4^{4n}}{3^{3n}} \cdot \frac{1}{\sqrt{\pi n}} \\
&= \left(\frac{256}{27}\right)^n \cdot \frac{1}{\sqrt{\pi n}}
\end{align}

Quindi \(\frac{1}{\binom{4n}{3n}} \sim \frac{\sqrt{\pi n}}{{(256/27)}^n}\), che converge rapidamente.
\end{esempio}

\begin{info}
\textbf{Serie con Funzioni Inverse:}
Per serie del tipo \(\sum f(\arctan(g(n)))\) o \(\sum \arcsin(h(n))\):
\begin{itemize}
    \item Se \(g(n) \to 0\): usa \(\arctan(x) \sim x\) per \(x \to 0\)
    \item Se \(g(n) \to \infty\): usa \(\arctan(x) \to \pi/2\) per \(x \to +\infty\)
    \item Per \(\arcsin\): attenzione al dominio \([-1,1]\)
\end{itemize}
\end{info}

\begin{info}
\textbf{Serie con Esponenziali ``Nascosti'':}
Per serie come \(\sum (e^{f(n)} - 1)\) con \(f(n) \to 0\):
\begin{itemize}
    \item Usa sempre \(e^x - 1 \sim x\) per \(x \to 0\)
    \item Per \(e^{1/n^2} - 1 \sim 1/n^2\): converge come serie armonica generalizzata
    \item Per \(e^{1/n} - 1 \sim 1/n\): diverge come serie armonica
\end{itemize}
\end{info}
