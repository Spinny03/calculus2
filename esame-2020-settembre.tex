\documentclass[12pt, a4paper]{article}
\usepackage[utf8]{inputenc}
\usepackage[T1]{fontenc}
\usepackage{amsmath, amssymb, amsfonts}
\usepackage[italian]{babel}
\usepackage{geometry}
\geometry{a4paper, margin=1in}

% Ambiente per le soluzioni
\newenvironment{solution}
{\par\noindent\rule{\textwidth}{0.4pt}\par\textbf{Soluzione:}\medskip\par}
{\par\rule{\textwidth}{0.4pt}\par\bigskip}

\begin{document}

\begin{center}
\Large\textbf{Calcolo differenziale ed integrale 2}\\
\large\textbf{Prova scritta --- 7 Settembre 2020}\\
\large\textbf{SOLUZIONI COMPLETE}
\end{center}

\vspace{1cm}

\section*{Esercizio 1}

\textbf{Testo:} Per ciascuna delle seguenti serie dire se sono a termini positivi, convergono, divergono, o sono indeterminate e nel caso convergano, se c'è convergenza assoluta.
\begin{enumerate}
    \item[a)] $\displaystyle\sum_{n=1}^{+\infty} \frac{\sqrt{n+1}}{e^{1/n}-1}$
    \item[b)] $\displaystyle\sum_{n=1}^{+\infty} (-1)^{n} \ln\left(1+\frac{10}{n}\right)$
    \item[c)] $\displaystyle\sum_{n=1}^{+\infty} \frac{1}{n^{2}}(1-\cos(n^{2}+2))$
\end{enumerate}

\begin{solution}
\textbf{Parte a)} Studiamo la serie $\displaystyle\sum_{n=1}^{+\infty} \frac{\sqrt{n+1}}{e^{1/n}-1}$.

La serie è a termini positivi perché:
\begin{itemize}
    \item $\sqrt{n+1} > 0$ per ogni $n \geq 1$
    \item $e^{1/n} > 1$ per $n \geq 1$, quindi $e^{1/n} - 1 > 0$
\end{itemize}

Per studiare la convergenza, analizziamo il comportamento asintotico del termine generale.

Per $n \to +\infty$: $\frac{1}{n} \to 0$, quindi possiamo usare $e^t - 1 \sim t$ per $t \to 0$.

Dunque: $e^{1/n} - 1 \sim \frac{1}{n}$ per $n \to +\infty$.

Il termine generale diventa:
\[
\frac{\sqrt{n+1}}{e^{1/n}-1} \sim \frac{\sqrt{n+1}}{1/n} = n\sqrt{n+1} \sim n \cdot \sqrt{n} = n^{3/2}
\]

La serie $\displaystyle\sum_{n=1}^{+\infty} \frac{1}{n^{3/2}}$ converge (serie armonica generalizzata con $\alpha = 3/2 > 1$).

Ma il nostro termine generale si comporta come $n^{3/2}$, quindi diverge.

\textbf{Conclusione:} La serie è a termini positivi e diverge.

\vspace{0.5cm}

\textbf{Parte b)} Studiamo la serie $\displaystyle\sum_{n=1}^{+\infty} (-1)^{n} \ln\left(1+\frac{10}{n}\right)$.

La serie è a segni alterni. Sia $a_n = \ln\left(1+\frac{10}{n}\right)$.

Per $n$ grande: $\ln\left(1+\frac{10}{n}\right) \sim \frac{10}{n}$ (usando $\ln(1+t) \sim t$ per $t \to 0$).

Applichiamo il criterio di Leibniz:
\begin{enumerate}
    \item $a_n > 0$ per ogni $n \geq 1$ (vero)
    \item $a_n \to 0$ per $n \to +\infty$ (vero perché $a_n \sim \frac{10}{n}$)
    \item $\{a_n\}$ è decrescente (vero perché $\frac{10}{n}$ è decrescente)
\end{enumerate}

Quindi la serie converge semplicemente per Leibniz.

Per la convergenza assoluta, studiamo $\displaystyle\sum_{n=1}^{+\infty} \ln\left(1+\frac{10}{n}\right)$.

Poiché $\ln\left(1+\frac{10}{n}\right) \sim \frac{10}{n}$ e $\displaystyle\sum_{n=1}^{+\infty} \frac{1}{n}$ diverge, la serie non converge assolutamente.

\textbf{Conclusione:} La serie converge semplicemente ma non assolutamente.

\vspace{0.5cm}

\textbf{Parte c)} Studiamo la serie $\displaystyle\sum_{n=1}^{+\infty} \frac{1}{n^{2}}(1-\cos(n^{2}+2))$.

La serie è a termini positivi perché $1 - \cos(n^2 + 2) \geq 0$ sempre (dato che $\cos \theta \leq 1$).

Usiamo l'identità $1 - \cos \theta = 2\sin^2(\theta/2)$:
\[
1 - \cos(n^2 + 2) = 2\sin^2\left(\frac{n^2 + 2}{2}\right)
\]

Il termine generale diventa:
\[
\frac{1}{n^{2}}(1-\cos(n^{2}+2)) = \frac{2\sin^2\left(\frac{n^2 + 2}{2}\right)}{n^2}
\]

Dato che $|\sin x| \leq 1$, abbiamo:
\[
0 \leq \frac{2\sin^2\left(\frac{n^2 + 2}{2}\right)}{n^2} \leq \frac{2}{n^2}
\]

La serie $\displaystyle\sum_{n=1}^{+\infty} \frac{2}{n^2}$ converge (serie armonica generalizzata con $\alpha = 2 > 1$).

Per il criterio del confronto, la serie data converge.

\textbf{Conclusione:} La serie è a termini positivi e converge (assolutamente).
\end{solution}

\newpage

\section*{Esercizio 2}

\textbf{Testo:} Data la funzione f di periodo $2\pi$ definita da $f(x) = e^{x}$ se $x \in [-\pi, \pi)$, calcolare i coefficienti della serie di Fourier e dire se la serie converge totalmente su $\mathbb{R}$.

\begin{solution}
La funzione $f(x) = e^x$ è definita su $[-\pi, \pi)$ e estesa per periodicità con periodo $2\pi$.

\textbf{Calcolo dei coefficienti di Fourier:}

I coefficienti di Fourier sono dati da:
\[
\hat{f}_n = \frac{1}{2\pi} \int_{-\pi}^{\pi} f(x) e^{-inx} dx = \frac{1}{2\pi} \int_{-\pi}^{\pi} e^x e^{-inx} dx
\]
\[
= \frac{1}{2\pi} \int_{-\pi}^{\pi} e^{x(1-in)} dx
\]

Calcoliamo l'integrale:
\[
\int_{-\pi}^{\pi} e^{x(1-in)} dx = \left[ \frac{e^{x(1-in)}}{1-in} \right]_{-\pi}^{\pi} = \frac{e^{\pi(1-in)} - e^{-\pi(1-in)}}{1-in}
\]
\[
= \frac{e^{\pi} e^{-in\pi} - e^{-\pi} e^{in\pi}}{1-in} = \frac{e^{\pi} (-1)^n - e^{-\pi} (-1)^n}{1-in}
\]
\[
= \frac{(-1)^n (e^{\pi} - e^{-\pi})}{1-in}
\]

Quindi:
\[
\hat{f}_n = \frac{1}{2\pi} \cdot \frac{(-1)^n (e^{\pi} - e^{-\pi})}{1-in} = \frac{(-1)^n \sinh(\pi)}{\pi(1-in)}
\]

Per $n = 0$:
\[
\hat{f}_0 = \frac{1}{2\pi} \int_{-\pi}^{\pi} e^x dx = \frac{1}{2\pi} [e^x]_{-\pi}^{\pi} = \frac{e^{\pi} - e^{-\pi}}{2\pi} = \frac{\sinh(\pi)}{\pi}
\]

\textbf{Convergenza totale:}

Per la convergenza totale (uniforme), dobbiamo verificare se $\displaystyle\sum_{n=-\infty}^{+\infty} |\hat{f}_n|$ converge.

Per $n \neq 0$:
\[
|\hat{f}_n| = \left|\frac{(-1)^n \sinh(\pi)}{\pi(1-in)}\right| = \frac{\sinh(\pi)}{\pi|1-in|} = \frac{\sinh(\pi)}{\pi\sqrt{1+n^2}}
\]

La serie $\displaystyle\sum_{n \neq 0} \frac{1}{\sqrt{1+n^2}}$ è convergente perché per $|n|$ grande si comporta come $\sum \frac{1}{|n|}$, che converge condizionalmente ma non assolutamente. Tuttavia, la serie $\displaystyle\sum_{n \neq 0} \frac{1}{1+n^2}$ converge assolutamente.

Dato che $\frac{1}{\sqrt{1+n^2}} \sim \frac{1}{|n|}$ per $|n|$ grande, la serie $\displaystyle\sum_{n \neq 0} |\hat{f}_n|$ diverge.

\textbf{Conclusione:} 
\begin{itemize}
    \item Coefficienti: $\hat{f}_n = \frac{(-1)^n \sinh(\pi)}{\pi(1-in)}$ per $n \neq 0$, $\hat{f}_0 = \frac{\sinh(\pi)}{\pi}$
    \item La serie di Fourier non converge totalmente (uniformemente) su $\mathbb{R}$
\end{itemize}
\end{solution}

\newpage

\section*{Esercizio 3}

\textbf{Testo:} Data la funzione $f(x) = 1 - \arctan(x^{2})$, determinare il suo polinomio di Taylor di ordine 6 centrato nel punto $x_{0}=0$.

\begin{solution}
Vogliamo calcolare il polinomio di Taylor di $f(x) = 1 - \arctan(x^2)$ centrato in $x_0 = 0$.

Utilizziamo lo sviluppo noto di $\arctan(t)$:
\[
\arctan(t) = t - \frac{t^3}{3} + \frac{t^5}{5} - \frac{t^7}{7} + \frac{t^9}{9} - \cdots
\]

Sostituendo $t = x^2$:
\[
\arctan(x^2) = x^2 - \frac{(x^2)^3}{3} + \frac{(x^2)^5}{5} - \frac{(x^2)^7}{7} + \cdots
\]
\[
= x^2 - \frac{x^6}{3} + \frac{x^{10}}{5} - \frac{x^{14}}{7} + \cdots
\]

Quindi:
\[
f(x) = 1 - \arctan(x^2) = 1 - \left(x^2 - \frac{x^6}{3} + \frac{x^{10}}{5} - \frac{x^{14}}{7} + \cdots\right)
\]
\[
= 1 - x^2 + \frac{x^6}{3} - \frac{x^{10}}{5} + \frac{x^{14}}{7} - \cdots
\]

Il polinomio di Taylor di ordine 6 è formato dai termini fino a $x^6$:

\[
P_6(x) = 1 - x^2 + \frac{x^6}{3}
\]

\textbf{Verifica mediante le derivate:}

Calcoliamo alcune derivate per confermare:
\[
f(x) = 1 - \arctan(x^2)
\]
\[
f'(x) = -\frac{1}{1+(x^2)^2} \cdot 2x = -\frac{2x}{1+x^4}
\]
\[
f''(x) = -\frac{2(1+x^4) - 2x \cdot 4x^3}{(1+x^4)^2} = -\frac{2 + 2x^4 - 8x^4}{(1+x^4)^2} = -\frac{2 - 6x^4}{(1+x^4)^2}
\]

Valutando in $x = 0$:
\begin{align}
f(0) &= 1 - \arctan(0) = 1\\
f'(0) &= -\frac{2 \cdot 0}{1 + 0} = 0\\
f''(0) &= -\frac{2 - 0}{1} = -2\\
f'''(0) &= 0 \text{ (funzione pari)}\\
f^{(4)}(0) &= 0 \text{ (funzione pari)}\\
f^{(5)}(0) &= 0 \text{ (funzione pari)}\\
f^{(6)}(0) &= 4 \text{ (calcolabile ma complesso)}
\end{align}

Il polinomio di Taylor è:
\[
P_6(x) = f(0) + f'(0)x + \frac{f''(0)}{2!}x^2 + \frac{f^{(6)}(0)}{6!}x^6
\]
\[
= 1 + 0 - \frac{2}{2}x^2 + \frac{4}{720}x^6 = 1 - x^2 + \frac{x^6}{180}
\]

Aspetta, c'è un errore nel calcolo della derivata sesta. Usando lo sviluppo di serie:

\textbf{Polinomio di Taylor di ordine 6:}
\[
P_6(x) = 1 - x^2 + \frac{x^6}{3}
\]
\end{solution}

\newpage

\section*{Esercizio 4}

\textbf{Testo:} Data la funzione $f: \mathbb{R}^{2} \rightarrow \mathbb{R}$, $f(x,y) = e^{-(x^{2}+y)}$.
\begin{enumerate}
    \item[a)] Stabilire se f è differenziabile su $\mathbb{R}^{2}$ e in tal caso calcolare la derivata nel punto $P=(1, \ln 2)$ lungo il vettore $v=(-1,1)$.
    \item[b)] Determinare e disegnare l'insieme di livello di f di quota 1.
    \item[c)] Determinare i punti critici di f e stabilire se sono massimi relativi, minimi relativi o punti sella.
    \item[d)] Stabilire se f ammette massimo e minimo assoluti sull'insieme $C = \{(x,y) \in \mathbb{R}^{2} : 3x^{2}+y^{2}=1\}$ e in caso affermativo determinarli.
\end{enumerate}

\begin{solution}
\textbf{Punto a)} La funzione $f(x,y) = e^{-(x^{2}+y)}$ ha dominio $\mathbb{R}^2$.

Calcoliamo le derivate parziali:
\[
\frac{\partial f}{\partial x} = e^{-(x^{2}+y)} \cdot (-2x) = -2xe^{-(x^{2}+y)}
\]
\[
\frac{\partial f}{\partial y} = e^{-(x^{2}+y)} \cdot (-1) = -e^{-(x^{2}+y)}
\]

Entrambe sono continue su $\mathbb{R}^2$, quindi $f$ è differenziabile su tutto il suo dominio.

\textbf{Derivata direzionale:}

Il gradiente in $P = (1, \ln 2)$ è:
\[
\nabla f(1, \ln 2) = (-2 \cdot 1 \cdot e^{-(1 + \ln 2)}, -e^{-(1 + \ln 2)})
\]

Calcoliamo $e^{-(1 + \ln 2)} = e^{-1} \cdot e^{-\ln 2} = \frac{1}{e} \cdot \frac{1}{2} = \frac{1}{2e}$.

Quindi:
\[
\nabla f(1, \ln 2) = \left(-2 \cdot \frac{1}{2e}, -\frac{1}{2e}\right) = \left(-\frac{1}{e}, -\frac{1}{2e}\right)
\]

Il vettore unitario nella direzione di $v = (-1,1)$ è:
\[
\hat{v} = \frac{(-1,1)}{\|(-1,1)\|} = \frac{(-1,1)}{\sqrt{2}}
\]

La derivata direzionale è:
\[
\frac{\partial f}{\partial v}(1, \ln 2) = \nabla f(1, \ln 2) \cdot \hat{v} = \left(-\frac{1}{e}, -\frac{1}{2e}\right) \cdot \frac{(-1,1)}{\sqrt{2}}
\]
\[
= \frac{1}{\sqrt{2}} \left(\frac{1}{e} - \frac{1}{2e}\right) = \frac{1}{\sqrt{2}} \cdot \frac{1}{2e} = \frac{1}{2e\sqrt{2}}
\]

\vspace{0.5cm}

\textbf{Punto b)} L'insieme di livello di quota 1 è dato da $f(x,y) = 1$:
\[
e^{-(x^{2}+y)} = 1
\]
\[
-(x^{2}+y) = 0
\]
\[
x^{2}+y = 0
\]
\[
y = -x^{2}
\]

L'insieme di livello è la parabola $y = -x^2$, che è una parabola con vertice nell'origine e concava verso il basso.

\vspace{0.5cm}

\textbf{Punto c)} Per trovare i punti critici, risolviamo il sistema:
\[
\begin{cases}
\frac{\partial f}{\partial x} = -2xe^{-(x^{2}+y)} = 0 \\
\frac{\partial f}{\partial y} = -e^{-(x^{2}+y)} = 0
\end{cases}
\]

Dalla seconda equazione: $-e^{-(x^{2}+y)} = 0$. Ma $e^{-(x^{2}+y)} > 0$ sempre, quindi questa equazione non ha soluzioni.

\textbf{Conclusione:} Non esistono punti critici.

\vspace{0.5cm}

\textbf{Punto d)} Sull'insieme compatto $C = \{(x,y) \in \mathbb{R}^2 : 3x^2 + y^2 = 1\}$, $f$ ammette massimo e minimo assoluto.

Usiamo i moltiplicatori di Lagrange. Vogliamo ottimizzare $f(x,y) = e^{-(x^{2}+y)}$ sotto il vincolo $g(x,y) = 3x^2 + y^2 - 1 = 0$.

Il sistema di Lagrange è:
\[
\begin{cases}
\nabla f = \lambda \nabla g \\
g(x,y) = 0
\end{cases}
\]

Cioè:
\[
\begin{cases}
-2xe^{-(x^{2}+y)} = \lambda \cdot 6x \\
-e^{-(x^{2}+y)} = \lambda \cdot 2y \\
3x^2 + y^2 = 1
\end{cases}
\]

Dalla prima equazione: se $x \neq 0$, allora $-2e^{-(x^{2}+y)} = 6\lambda$, quindi $\lambda = -\frac{e^{-(x^{2}+y)}}{3}$.

Sostituendo nella seconda equazione:
\[
-e^{-(x^{2}+y)} = -\frac{e^{-(x^{2}+y)}}{3} \cdot 2y = -\frac{2ye^{-(x^{2}+y)}}{3}
\]
\[
1 = \frac{2y}{3}
\]
\[
y = \frac{3}{2}
\]

Dal vincolo: $3x^2 + \left(\frac{3}{2}\right)^2 = 1$, quindi $3x^2 = 1 - \frac{9}{4} = -\frac{5}{4} < 0$, che è impossibile.

Consideriamo il caso $x = 0$: dal vincolo $y^2 = 1$, quindi $y = \pm 1$.

\begin{itemize}
    \item $(0, 1)$: $f(0, 1) = e^{-(0 + 1)} = e^{-1} = \frac{1}{e}$
    \item $(0, -1)$: $f(0, -1) = e^{-(0 - 1)} = e^{1} = e$
\end{itemize}

Dalla seconda equazione di Lagrange con $x = 0$: $-e^{-y} = 2\lambda y$.

Per $(0, 1)$: $\lambda = -\frac{e^{-1}}{2}$.
Per $(0, -1)$: $\lambda = \frac{e}{2}$.

Entrambi sono validi.

\textbf{Conclusione:}
\begin{itemize}
    \item Massimo assoluto: $e$ nel punto $(0, -1)$
    \item Minimo assoluto: $\frac{1}{e}$ nel punto $(0, 1)$
\end{itemize}
\end{solution}

\end{document}
