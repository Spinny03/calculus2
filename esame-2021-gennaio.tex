\documentclass[12pt, a4paper]{article}
\usepackage[utf8]{inputenc}
\usepackage[T1]{fontenc}
\usepackage{amsmath, amssymb, amsfonts}
\usepackage[italian]{babel}
\usepackage{geometry}
\geometry{a4paper, margin=1in}

% Ambiente per le soluzioni
\newenvironment{solution}
{\par\noindent\rule{\textwidth}{0.4pt}\par\textbf{Soluzione:}\medskip\par}
{\par\rule{\textwidth}{0.4pt}\par\bigskip}

\begin{document}

\begin{center}
\Large\textbf{Calcolo differenziale ed integrale 2}\\
\large\textbf{Prova scritta --- 14 Gennaio 2021}\\
\large\textbf{SOLUZIONI }
\end{center}

\vspace{1cm}

\section*{Esercizio 1}

\textbf{Testo:} Per ciascuna delle seguenti serie dire se sono a valori positivi, convergono semplicemente, convergono assolutamente.
\begin{enumerate}
    \item $\displaystyle\sum_{n=1}^{+\infty} \frac{\cos(n^{4})}{n^{2}+1}$
    \item $\displaystyle\sum_{n=1}^{+\infty} (-1)^{n} \frac{3^{n}}{4^{n}+n^{2}}$
\end{enumerate}
Stabilire inoltre se la ridotta $s_{9}$ della serie (2) approssima il valore della serie a meno di 0,1.
\begin{enumerate}
    \item[3.] Determinare il raggio e l'intervallo di convergenza della serie di potenze $\displaystyle\sum_{n=1}^{+\infty} \frac{1}{\sqrt{n}}x^{n}$.
\end{enumerate}

\begin{solution}
\textbf{Parte 1)} Studiamo la serie $\displaystyle\sum_{n=1}^{+\infty} \frac{\cos(n^{4})}{n^{2}+1}$.

La serie non è a valori positivi perché $\cos(n^4)$ può essere negativo.

Per la convergenza assoluta, consideriamo $\left|\frac{\cos(n^{4})}{n^{2}+1}\right| \leq \frac{1}{n^{2}+1}$.

La serie $\displaystyle\sum_{n=1}^{+\infty} \frac{1}{n^{2}+1}$ è convergente perché si comporta come $\displaystyle\sum_{n=1}^{+\infty} \frac{1}{n^2}$ (serie armonica generalizzata con $\alpha = 2 > 1$).

Per il criterio del confronto, la serie data converge assolutamente (e quindi anche semplicemente).

\textbf{Conclusione:} Non è a valori positivi, converge assolutamente.

\vspace{0.5cm}

\textbf{Parte 2)} Studiamo la serie $\displaystyle\sum_{n=1}^{+\infty} (-1)^{n} \frac{3^{n}}{4^{n}+n^{2}}$.

La serie è a segni alterni. Sia $a_n = \frac{3^{n}}{4^{n}+n^{2}}$.

Per $n$ grande: $4^n \gg n^2$, quindi $a_n \sim \frac{3^{n}}{4^{n}} = \left(\frac{3}{4}\right)^n$.

Applichiamo il criterio di Leibniz:
\begin{enumerate}
    \item $a_n > 0$ per ogni $n \geq 1$ (vero)
    \item $a_n \to 0$ per $n \to +\infty$ (vero perché $\left(\frac{3}{4}\right)^n \to 0$)
    \item $\{a_n\}$ è decrescente (vero per $n$ sufficientemente grande)
\end{enumerate}

Quindi la serie converge semplicemente per Leibniz.

Per la convergenza assoluta, studiamo $\displaystyle\sum_{n=1}^{+\infty} \frac{3^{n}}{4^{n}+n^{2}}$.

Poiché $\frac{3^{n}}{4^{n}+n^{2}} \sim \left(\frac{3}{4}\right)^n$ e $\displaystyle\sum_{n=1}^{+\infty} \left(\frac{3}{4}\right)^n$ converge (serie geometrica con $q = 3/4 < 1$), la serie converge anche assolutamente.

\textbf{Stima dell'errore per $s_9$:}

Per una serie a segni alterni convergente, l'errore è limitato dal primo termine trascurato:
\[
|S - s_9| \leq a_{10} = \frac{3^{10}}{4^{10}+10^{2}} = \frac{59049}{1048576+100} = \frac{59049}{1048676} \approx 0.0563 < 0.1
\]

Quindi $s_9$ approssima la somma a meno di 0,1.

\textbf{Conclusione:} Non è a valori positivi, converge assolutamente, $s_9$ è una buona approssimazione.

\vspace{0.5cm}

\textbf{Parte 3)} Studiamo la serie di potenze $\displaystyle\sum_{n=1}^{+\infty} \frac{1}{\sqrt{n}}x^{n}$.

Il coefficiente è $a_n = \frac{1}{\sqrt{n}}$.

Calcoliamo il raggio di convergenza usando il criterio della radice:
\[
\rho = \frac{1}{\limsup_{n \to +\infty} \sqrt[n]{|a_n|}} = \frac{1}{\limsup_{n \to +\infty} \sqrt[n]{\frac{1}{\sqrt{n}}}}
\]

Poiché $\sqrt[n]{\frac{1}{\sqrt{n}}} = \frac{1}{n^{1/(2n)}}$ e $\lim_{n \to +\infty} n^{1/(2n)} = 1$, abbiamo $\rho = 1$.

Studio dei punti estremi:
\begin{itemize}
    \item Per $x = 1$: $\displaystyle\sum_{n=1}^{+\infty} \frac{1}{\sqrt{n}}$ diverge (serie armonica generalizzata con $\alpha = 1/2 < 1$)
    \item Per $x = -1$: $\displaystyle\sum_{n=1}^{+\infty} \frac{(-1)^n}{\sqrt{n}}$ converge per Leibniz
\end{itemize}

\textbf{Conclusione:} Raggio di convergenza $\rho = 1$, intervallo di convergenza $[-1, 1)$.
\end{solution}

\newpage

\section*{Esercizio 2}

\textbf{Testo:} Determinare il polinomio di Taylor centrato in 0 e di ordine 4 di $f(x) = \log(\cos x)$.

\begin{solution}
Vogliamo calcolare il polinomio di Taylor di $f(x) = \log(\cos x)$ centrato in $x_0 = 0$.

Utilizziamo gli sviluppi noti:
\[
\cos x = 1 - \frac{x^2}{2} + \frac{x^4}{24} - \frac{x^6}{720} + O(x^8)
\]

Quindi:
\[
\cos x = 1 + \left(-\frac{x^2}{2} + \frac{x^4}{24} + O(x^6)\right)
\]

Sia $u = -\frac{x^2}{2} + \frac{x^4}{24} + O(x^6)$. Allora:
\[
\log(\cos x) = \log(1 + u)
\]

Usando lo sviluppo $\log(1 + u) = u - \frac{u^2}{2} + \frac{u^3}{3} - \frac{u^4}{4} + O(u^5)$:

\[
u = -\frac{x^2}{2} + \frac{x^4}{24}
\]
\[
u^2 = \left(-\frac{x^2}{2}\right)^2 + O(x^6) = \frac{x^4}{4} + O(x^6)
\]
\[
u^3 = O(x^6), \quad u^4 = O(x^8)
\]

Quindi:
\[
\log(\cos x) = u - \frac{u^2}{2} + O(x^6)
\]
\[
= \left(-\frac{x^2}{2} + \frac{x^4}{24}\right) - \frac{1}{2} \cdot \frac{x^4}{4} + O(x^6)
\]
\[
= -\frac{x^2}{2} + \frac{x^4}{24} - \frac{x^4}{8} + O(x^6)
\]
\[
= -\frac{x^2}{2} + x^4\left(\frac{1}{24} - \frac{1}{8}\right) + O(x^6)
\]
\[
= -\frac{x^2}{2} + x^4\left(\frac{1 - 3}{24}\right) + O(x^6)
\]
\[
= -\frac{x^2}{2} - \frac{x^4}{12} + O(x^6)
\]

\textbf{Polinomio di Taylor di ordine 4:}
\[
P_4(x) = -\frac{x^2}{2} - \frac{x^4}{12}
\]

\textbf{Verifica:} $f(0) = \log(\cos 0) = \log(1) = 0$, $f'(0) = \frac{-\sin(0)}{\cos(0)} = 0$, che è coerente con il nostro risultato.
\end{solution}

\newpage

\section*{Esercizio 3}

\textbf{Testo:} Data la funzione $f: \mathbb{R} \rightarrow \mathbb{R}$ periodica di periodo 4 definita da
\[ f(x) = \begin{cases} 0 & -2 \le x < 1 \\ -1 & 1 \le x < 2 \end{cases} \]
determinare i suoi coefficienti di Fourier $\hat{f}_{2n+1}$ per $n \in \mathbb{Z}$. Determinare poi il valore della serie di Fourier di f in $x=0$ e $x=1$.

\begin{solution}
La funzione ha periodo $T = 4$, quindi $\omega = \frac{2\pi}{T} = \frac{\pi}{2}$.

I coefficienti di Fourier complessi sono:
\[
\hat{f}_k = \frac{1}{4} \int_{-2}^{2} f(x) e^{-i k \frac{\pi}{2} x} dx
\]

\[
= \frac{1}{4} \left[\int_{-2}^{1} 0 \cdot e^{-i k \frac{\pi}{2} x} dx + \int_{1}^{2} (-1) e^{-i k \frac{\pi}{2} x} dx\right]
\]
\[
= -\frac{1}{4} \int_{1}^{2} e^{-i k \frac{\pi}{2} x} dx
\]

Per $k \neq 0$:
\[
\int_{1}^{2} e^{-i k \frac{\pi}{2} x} dx = \left[\frac{e^{-i k \frac{\pi}{2} x}}{-i k \frac{\pi}{2}}\right]_{1}^{2} = \frac{2}{-i k \pi}\left[e^{-i k \pi} - e^{-i k \frac{\pi}{2}}\right]
\]
\[
= \frac{2}{-i k \pi}\left[(-1)^k - e^{-i k \frac{\pi}{2}}\right]
\]

Per $k = 2n+1$ (dispari):
\[
e^{-i (2n+1) \frac{\pi}{2}} = e^{-i \pi n} e^{-i \frac{\pi}{2}} = (-1)^n (-i) = (-1)^{n+1} i
\]

Quindi:
\[
\hat{f}_{2n+1} = -\frac{1}{4} \cdot \frac{2}{-i (2n+1) \pi}\left[(-1)^{2n+1} - (-1)^{n+1} i\right]
\]
\[
= \frac{1}{2i (2n+1) \pi}\left[-1 - (-1)^{n+1} i\right]
\]
\[
= \frac{1}{2i (2n+1) \pi}\left[-1 + (-1)^{n} i\right]
\]

Per $k = 0$:
\[
\hat{f}_0 = -\frac{1}{4} \int_{1}^{2} 1 dx = -\frac{1}{4}
\]

\textbf{Convergenza della serie:}

In $x = 0$: la funzione è continua e $f(0) = 0$, quindi la serie converge a 0.

In $x = 1$: c'è una discontinuità di salto:
\[
\lim_{x \to 1^-} f(x) = 0, \quad \lim_{x \to 1^+} f(x) = -1
\]

La serie converge alla media: $\frac{0 + (-1)}{2} = -\frac{1}{2}$.

\textbf{Conclusione:}
\begin{itemize}
    \item $\hat{f}_{2n+1} = \frac{1}{2i (2n+1) \pi}\left[-1 + (-1)^{n} i\right]$
    \item In $x = 0$: serie converge a 0
    \item In $x = 1$: serie converge a $-\frac{1}{2}$
\end{itemize}
\end{solution}

\newpage

\section*{Esercizio 4}

\textbf{Testo:} Data la funzione $f: D \subset \mathbb{R}^{2} \rightarrow \mathbb{R}$, $f(x,y) = \frac{1}{3-xy}$.
\begin{enumerate}
    \item[a)] Trovare il dominio di f e specificarne le caratteristiche: dire se è aperto, limitato, connesso.
    \item[b)] Stabilire se f è differenziabile su D e in tal caso calcolare il piano tangente al grafico di f nel punto $(0,0,f(0,0))$.
    \item[c)] Determinare i punti critici di f.
    \item[d)] Stabilire se f ammette massimo e minimo assoluti sull'insieme $C = \{(x,y) \in \mathbb{R}^{2} : x^{2}+y^{2}=2\}$ e in caso affermativo determinarli.
\end{enumerate}

\begin{solution}
\textbf{Punto a)} Il dominio di $f(x,y) = \frac{1}{3-xy}$ è determinato dalla condizione $3 - xy \neq 0$, cioè $xy \neq 3$.

Quindi: $D = \{(x,y) \in \mathbb{R}^2 : xy \neq 3\}$.

L'insieme $\{(x,y) : xy = 3\}$ è l'iperbole $xy = 3$. Il dominio D è il complemento di questa iperbole.

\textbf{Caratteristiche di D:}
\begin{itemize}
    \item \textbf{Aperto:} Sì, perché è il complemento di un insieme chiuso (l'iperbole)
    \item \textbf{Limitato:} No, contiene punti arbitrariamente lontani dall'origine
    \item \textbf{Connesso:} No, l'iperbole $xy = 3$ separa il piano in più componenti connesse
\end{itemize}

\vspace{0.5cm}

\textbf{Punto b)} Calcoliamo le derivate parziali:
\[
\frac{\partial f}{\partial x} = \frac{y}{(3-xy)^2}
\]
\[
\frac{\partial f}{\partial y} = \frac{x}{(3-xy)^2}
\]

Entrambe sono continue su D, quindi f è differenziabile su D.

\textbf{Piano tangente in $(0,0,f(0,0))$:}

$f(0,0) = \frac{1}{3-0} = \frac{1}{3}$.

Le derivate parziali nel punto $(0,0)$:
\[
\frac{\partial f}{\partial x}(0,0) = \frac{0}{(3-0)^2} = 0
\]
\[
\frac{\partial f}{\partial y}(0,0) = \frac{0}{(3-0)^2} = 0
\]

L'equazione del piano tangente in $(0,0,\frac{1}{3})$ è:
\[
z - \frac{1}{3} = 0(x - 0) + 0(y - 0)
\]
\[
z = \frac{1}{3}
\]

\vspace{0.5cm}

\textbf{Punto c)} Per trovare i punti critici, risolviamo il sistema:
\[
\begin{cases}
\frac{\partial f}{\partial x} = \frac{y}{(3-xy)^2} = 0 \\
\frac{\partial f}{\partial y} = \frac{x}{(3-xy)^2} = 0
\end{cases}
\]

Dalla prima equazione: $y = 0$ (poiché $3 - xy \neq 0$ su D).
Dalla seconda equazione: $x = 0$.

L'unico punto critico è $(0, 0)$.

\vspace{0.5cm}

\textbf{Punto d)} Sull'insieme $C = \{(x,y) \in \mathbb{R}^2 : x^2 + y^2 = 2\}$, verifichiamo prima che C sia contenuto in D.

Per punti su C: $x^2 + y^2 = 2$. Per il teorema delle medie aritmetiche e geometriche:
\[
|xy| \leq \frac{x^2 + y^2}{2} = \frac{2}{2} = 1 < 3
\]

Quindi $xy \neq 3$ per tutti i punti di C, e C è contenuto in D.

Usiamo i moltiplicatori di Lagrange per ottimizzare $f(x,y) = \frac{1}{3-xy}$ sotto il vincolo $g(x,y) = x^2 + y^2 - 2 = 0$.

Il sistema è:
\[
\begin{cases}
\frac{y}{(3-xy)^2} = 2\lambda x \\
\frac{x}{(3-xy)^2} = 2\lambda y \\
x^2 + y^2 = 2
\end{cases}
\]

Se $x, y \neq 0$, dividendo la prima equazione per la seconda:
\[
\frac{y}{x} = \frac{x}{y} \Rightarrow y^2 = x^2 \Rightarrow y = \pm x
\]

\textbf{Caso 1:} $y = x$, allora $2x^2 = 2$, quindi $x = \pm 1$.
\begin{itemize}
    \item $(1, 1)$: $f(1, 1) = \frac{1}{3-1} = \frac{1}{2}$
    \item $(-1, -1)$: $f(-1, -1) = \frac{1}{3-1} = \frac{1}{2}$
\end{itemize}

\textbf{Caso 2:} $y = -x$, allora $2x^2 = 2$, quindi $x = \pm 1$.
\begin{itemize}
    \item $(1, -1)$: $f(1, -1) = \frac{1}{3+1} = \frac{1}{4}$
    \item $(-1, 1)$: $f(-1, 1) = \frac{1}{3+1} = \frac{1}{4}$
\end{itemize}

\textbf{Casi $x = 0$ o $y = 0$:} Dal vincolo otteniamo $(0, \pm\sqrt{2})$ e $(\pm\sqrt{2}, 0)$.
In tutti questi punti: $f = \frac{1}{3-0} = \frac{1}{3}$.

\textbf{Conclusione:}
\begin{itemize}
    \item Massimo assoluto: $\frac{1}{2}$ nei punti $(1, 1)$ e $(-1, -1)$
    \item Minimo assoluto: $\frac{1}{4}$ nei punti $(1, -1)$ e $(-1, 1)$
\end{itemize}
\end{solution}

\end{document}
