\section{Esercizio 2: Polinomi di Taylor/MacLaurin e Serie di Fourier}

\subsection{Polinomi di Taylor / MacLaurin - Tecniche Avanzate}
\begin{strategia}
Non calcolare \textbf{mai} le derivate una per una, a meno che non sia esplicitamente richiesto o la funzione sia banalissima. La strada maestra è usare gli \textbf{sviluppi notevoli} (vedi formulario) e combinarli algebricamente (somma, prodotto, composizione).
\end{strategia}

\subsubsection{Tecniche Standard}
\begin{itemize}
    \item \textbf{Come operare:} Sostituisci, somma, moltiplica e componi gli sviluppi notevoli. Ricorda di fermarti all'ordine richiesto e di usare il simbolo di o-piccolo \(o((x-x_0)^n)\). Gestisci con cura le potenze e gli ordini degli o-piccoli.
    \item \textbf{Calcolo di \(f^{(n)}(0)\):} Dalla teoria, il coefficiente \(c_n\) del termine \(x^n\) nello sviluppo di MacLaurin è \(c_n = \frac{f^{(n)}(0)}{n!}\). La formula inversa è potentissima per calcolare derivate in zero senza fatica: 
    \[ f^{(n)}(0) = n! \cdot c_n \]
    \item \textbf{Resto di Lagrange e Stima dell'Errore:} Il resto di Lagrange di ordine \(n\) è:
    \[ R_n(x) = \frac{f^{(n+1)}(\xi)}{(n+1)!}(x-x_0)^{n+1} \]
    dove \(\xi\) è un punto compreso tra \(x_0\) e \(x\). Per stimare l'errore:
    \begin{enumerate}
        \item Calcola la derivata \((n+1)\)-esima di \(f\).
        \item Trova il massimo di \(|f^{(n+1)}(\xi)|\) nell'intervallo di interesse.
        \item Applica la formula: \(|R_n(x)| \le \frac{M}{(n+1)!}|x-x_0|^{n+1}\) dove \(M = \max |f^{(n+1)}(\xi)|\).
    \end{enumerate}
\end{itemize}

\begin{esempio}[Esame 14 Gennaio 2021]
Determinare il polinomio di Taylor centrato in 0 e di ordine 4 di \(f(x) = \log(\cos x)\).

\textbf{Soluzione:} Usiamo gli sviluppi notevoli invece di calcolare le derivate:
\begin{align}
\cos x &= 1 - \frac{x^2}{2!} + \frac{x^4}{4!} + o(x^4) = 1 - \frac{x^2}{2} + \frac{x^4}{24} + o(x^4)
\end{align}

Per \(\log(1+t)\) con \(t = \cos x - 1 = -\frac{x^2}{2} + \frac{x^4}{24} + o(x^4)\):
\begin{align}
\log(1+t) &= t - \frac{t^2}{2} + \frac{t^3}{3} - \frac{t^4}{4} + o(t^4)
\end{align}

Sostituiamo \(t\):
\begin{align}
t &= -\frac{x^2}{2} + \frac{x^4}{24} + o(x^4) \\
t^2 &= \left(-\frac{x^2}{2}\right)^2 + o(x^4) = \frac{x^4}{4} + o(x^4) \\
t^3 &= o(x^4), \quad t^4 = o(x^4)
\end{align}

Quindi:
\begin{align}
\log(\cos x) &= \left(-\frac{x^2}{2} + \frac{x^4}{24}\right) - \frac{1}{2} \cdot \frac{x^4}{4} + o(x^4) \\
&= -\frac{x^2}{2} + \frac{x^4}{24} - \frac{x^4}{8} + o(x^4) \\
&= -\frac{x^2}{2} + \frac{x^4 - 3x^4}{24} + o(x^4) \\
&= -\frac{x^2}{2} - \frac{x^4}{12} + o(x^4)
\end{align}

Il polinomio di Taylor di ordine 4 è: \(T_4(x) = -\frac{x^2}{2} - \frac{x^4}{12}\)
\end{esempio}

\subsubsection{Tecniche Avanzate dalle Esercitazioni}

\begin{info}
\textbf{Calcolo di Derivate di Ordine Elevato:}
Per calcolare $f^{(n)}(0)$ con $n$ molto grande (es. $f^{(99)}(0)$), analizza la struttura della serie di Taylor:
\begin{itemize}
    \item Se la funzione ha uno sviluppo polinomiale di grado finito, le derivate di ordine superiore sono nulle.
    \item Per funzioni trigonometriche, sfrutta la periodicità delle derivate.
    \item Per funzioni esponenziali composte, usa la regola della derivazione a catena ripetuta.
\end{itemize}
\end{info}

\begin{esempio}
Calcolare $f^{(99)}(0)$ per $f(x) = x^2 + 2x + e^x - x\sin(3x)$.

Analizziamo termine per termine:
\begin{itemize}
    \item $x^2 + 2x$: polinomio di grado 2, quindi $\frac{d^{99}}{dx^{99}}(x^2 + 2x) = 0$
    \item $e^x$: ha $\frac{d^{99}}{dx^{99}}e^x = e^x$, quindi contribuisce 1 in $x=0$
    \item $x\sin(3x)$: sviluppando $\sin(3x) = 3x - \frac{(3x)^3}{3!} + \frac{(3x)^5}{5!} - \cdots$, otteniamo $x\sin(3x) = 3x^2 - \frac{3^3x^4}{3!} + \cdots$. Il termine di grado 99 è nullo.
\end{itemize}
Quindi $f^{(99)}(0) = 99! \cdot 0 + 1 + 0 = 1$.
\end{esempio}

\begin{esempio}
Dalle esercitazioni: $f(x) = (e^{3x}-1)\cos(2x) - x$

Per il polinomio di Taylor di ordine 4:
\begin{enumerate}
    \item $e^{3x} - 1 = 3x + \frac{(3x)^2}{2!} + \frac{(3x)^3}{3!} + \frac{(3x)^4}{4!} + o(x^4)$
    \item $\cos(2x) = 1 - \frac{(2x)^2}{2!} + \frac{(2x)^4}{4!} + o(x^4)$
    \item Moltiplicando e raccogliendo fino al quarto ordine:
    \[ f(x) = 2x^2 + \frac{7x^3}{3} - \frac{5x^4}{3} + o(x^4) \]
\end{enumerate}
\end{esempio}

\subsubsection{Stime di Precisione Numerica}

\begin{strategia}
Per stimare valori numerici (es. $\log(4/3)$ a meno di $10^{-5}$):
\begin{enumerate}
    \item Trova una funzione opportuna da sviluppare (es. $f(x) = \log(\cos x)$ per $x = \arccos(3/4)$).
    \item Sviluppa la funzione fino all'ordine necessario per raggiungere la precisione richiesta.
    \item Usa la stima del resto per verificare che l'errore sia inferiore alla soglia.
\end{enumerate}
\end{strategia}

\begin{esempio}
Dalle esercitazioni: stimare $\log(4/3)$ usando $f(x) = \log(\cos x)$.

Osserviamo che $\cos(\arccos(3/4)) = 3/4$, quindi $4/3 = 1/\cos(\arccos(3/4))$.
Tuttavia, è più semplice usare $\log(4/3) = \log(4) - \log(3)$ e sviluppi noti.
\end{esempio}

\subsection{Serie di Fourier - Tecniche Avanzate}
\begin{strategia}
\begin{enumerate}
    \item \textbf{Disegna la funzione!} Un grafico del prolungamento periodico ti aiuta a vedere subito simmetrie e punti di discontinuità. 
    \item \textbf{Simmetrie:} Se $f$ è \textbf{pari} ($f(-x)=f(x)$), allora tutti i $b_n=0$. Se $f$ è \textbf{dispari} ($f(-x)=-f(x)$), allora $a_0=0$ e tutti gli $a_n=0$. Questo ti dimezza il lavoro! 
    \item \textbf{Calcolo Coefficienti:} Usa le formule, prestando attenzione agli estremi di integrazione (di solito $[-\pi, \pi]$ o $[-T/2, T/2]$). L'integrazione per parti è quasi sempre necessaria. 
    \item \textbf{Convergenza Puntuale (Teorema di Dirichlet):}
        \begin{itemize}
            \item Dove $f$ è continua, la serie converge a $f(x)$. 
            \item Nei punti di discontinuità a salto $x_d$, la serie converge al valore medio del salto: $\frac{f(x_d^+) + f(x_d^-)}{2}$. 
        \end{itemize}
    \item \textbf{Identità di Parseval:} Utile per calcolare la somma di serie numeriche. $\frac{1}{T} \int_{-T/2}^{T/2} |f(x)|^2 dx = \frac{a_0^2}{4} + \frac{1}{2} \sum_{n=1}^\infty (a_n^2 + b_n^2)$. 
\end{enumerate}
\end{strategia}

\subsubsection{Tecniche Avanzate dalle Esercitazioni}

\begin{info}
\textbf{Serie di Fourier di Funzioni Traslate:}
Se $g(x) = f(x-a)$, allora i coefficienti di Fourier di $g$ sono:
\[ \hat{g}_n = e^{-ina\frac{2\pi}{T}} \hat{f}_n \]
Questo è utile per evitare di ricalcolare integrali complessi.
\end{info}

\begin{esempio}
Dalle esercitazioni: $g(x) = f(x-1)$ dove $f$ ha coefficienti $\hat{f}_n$.
Per $T = 2\pi$: $\hat{g}_2 = e^{-i \cdot 2 \cdot 1} \hat{f}_2 = e^{-2i} \hat{f}_2$
\end{esempio}

\begin{info}
\textbf{Prolungamento Pari/Dispari:}
Per estendere una funzione definita su $[0,L]$ a tutto l'asse:
\begin{itemize}
    \item \textbf{Prolungamento pari:} $f(-x) = f(x)$ produce solo termini in coseno
    \item \textbf{Prolungamento dispari:} $f(-x) = -f(x)$ produce solo termini in seno
\end{itemize}
Questa tecnica può semplificare notevolmente i calcoli.
\end{info}

\begin{esempio}[Funzione con parametro \(\alpha\) dalle esercitazioni]
Data la funzione di periodo \(2\pi\):
\[ g_\alpha(x) = \begin{cases} x^2 & x \in [0,\pi] \\ \alpha x & x \in (\pi,2\pi) \end{cases} \]

\textbf{1. Calcolo dei coefficienti di Fourier in funzione di \(\alpha\):}

\textbf{Coefficiente \(a_0\):}
\begin{align}
a_0 &= \frac{1}{\pi} \int_0^{2\pi} g_\alpha(x) \, dx \\
&= \frac{1}{\pi} \left[ \int_0^\pi x^2 \, dx + \int_\pi^{2\pi} \alpha x \, dx \right] \\
&= \frac{1}{\pi} \left[ \frac{x^3}{3} \Big|_0^\pi + \alpha \frac{x^2}{2} \Big|_\pi^{2\pi} \right] \\
&= \frac{\pi^2}{3} + \frac{3\alpha\pi}{2}
\end{align}

\textbf{Coefficienti \(a_n\) per \(n \geq 1\):}
\begin{align}
a_n &= \frac{1}{\pi} \int_0^{2\pi} g_\alpha(x) \cos(nx) \, dx \\
&= \frac{1}{\pi} \left[ \int_0^\pi x^2 \cos(nx) \, dx + \int_\pi^{2\pi} \alpha x \cos(nx) \, dx \right]
\end{align}

Per \(\int_0^\pi x^2 \cos(nx) \, dx\), usiamo integrazione per parti due volte:
\begin{align}
\int x^2 \cos(nx) \, dx &= \frac{x^2 \sin(nx)}{n} + \frac{2x \cos(nx)}{n^2} - \frac{2 \sin(nx)}{n^3}
\end{align}

Calcolando:
\begin{align}
\int_0^\pi x^2 \cos(nx) \, dx &= \frac{\pi^2 \sin(n\pi)}{n} + \frac{2\pi \cos(n\pi)}{n^2} - \frac{2 \sin(n\pi)}{n^3} \\
&= \frac{2\pi \cos(n\pi)}{n^2} = \frac{2\pi (-1)^n}{n^2}
\end{align}

Calcolando:
\begin{align}
\int_\pi^{2\pi} x \cos(nx) \, dx &= \left[ \frac{x \sin(nx)}{n} + \frac{\cos(nx)}{n^2} \right]_\pi^{2\pi} \\
&= \frac{2\pi \sin(2n\pi) - \pi \sin(n\pi)}{n} + \frac{\cos(2n\pi) - \cos(n\pi)}{n^2} \\
&= \frac{1 - (-1)^n}{n^2}
\end{align}
\end{esempio}
\begin{esempio}
Quindi:
\[ a_n = \frac{1}{\pi} \left[ \frac{2\pi (-1)^n}{n^2} + \alpha \frac{1 - (-1)^n}{n^2} \right] = \frac{2(-1)^n + \alpha(1 - (-1)^n)/\pi}{n^2} \]

\textbf{Coefficienti \(b_n\):}
Con calcoli analoghi per l'integrazione per parti:
\[ b_n = \frac{-2(-1)^n + \alpha((-1)^n - 1)}{n} \]

\textbf{2. Analisi della convergenza puntuale nei punti di discontinuità:}

Nel punto \(x = \pi\):
\begin{itemize}
    \item Limite sinistro: \(g_\alpha(\pi^-) = \pi^2\)
    \item Limite destro: \(g_\alpha(\pi^+) = \alpha\pi\)
\end{itemize}

La funzione ha una discontinuità a salto se \(\alpha \neq \pi\).

\textbf{3. Valore di convergenza della serie:}

Per il teorema di convergenza di Fourier, nel punto di discontinuità \(x = \pi\) la serie converge a:
\[ \frac{g_\alpha(\pi^-) + g_\alpha(\pi^+)}{2} = \frac{\pi^2 + \alpha\pi}{2} \]

\textbf{Casi particolari:}
\begin{itemize}
    \item Se \(\alpha = \pi\): la funzione è continua in \(x = \pi\) e la serie converge a \(g_\alpha(\pi) = \pi^2\)
    \item Se \(\alpha = 0\): discontinuità massima, serie converge a \(\frac{\pi^2}{2}\)
    \item Se \(\alpha < 0\): la funzione ha un salto verso il basso
\end{itemize}

\textbf{Studio della convergenza totale:}
La serie converge totalmente se \(\sum_{n=1}^{\infty} (|a_n| + |b_n|) < \infty\).
Dato che \(|a_n| \sim \frac{1}{n^2}\) e \(|b_n| \sim \frac{1}{n}\), la serie non converge totalmente per nessun valore di \(\alpha\) a causa dei termini \(b_n\).
\end{esempio}

\subsubsection{Convergenza Totale per Serie di Fourier}

\begin{info}
\textbf{Convergenza Totale:}
Una serie di Fourier converge totalmente su $\mathbb{R}$ se $\sum_{n=1}^{\infty} (|a_n| + |b_n|) < \infty$.

\textbf{Estensione di Funzioni per Convergenza Totale:}
Come suggerito nelle esercitazioni, per ottenere convergenza totale si può:
\begin{enumerate}
    \item Estendere la funzione a una funzione pari su un intervallo più ampio
    \item Cambiare la scala temporale (dilatazione/contrazione)
    \item Usare prolungamenti regolari che riducono le discontinuità
\end{enumerate}
\end{info}

\begin{esempio}
Dalle esercitazioni: per $g_\pi(x)$ definita su $[0,2\pi]$, estenderla a una funzione pari su $[-2\pi,2\pi]$ per ottenere convergenza totale della serie
\[ \sum_{n=-\infty}^{+\infty} c_n e^{-i\frac{nx}{2}} \]
\end{esempio}

\subsubsection{Casi Speciali delle Esercitazioni}

\begin{info}
\textbf{Funzioni con Discontinuità Multiple:}
Per funzioni come:
\[ f(x) = \begin{cases} -1 & x \in [-\pi,-\pi/2) \cup (\pi/2,\pi) \\ 1 & x \in (-\pi/2,\pi/2) \\ 0 & x = \pm\pi/2 \end{cases} \]

La convergenza della serie di Fourier:
\begin{itemize}
    \item Nei punti di continuità: converge a $f(x)$
    \item Nei punti di discontinuità: converge al valore medio
    \item Nei punti assegnati esplicitamente: converge al valore assegnato se coincide con il valore medio
\end{itemize}
\end{info}

\begin{strategia}
\textbf{Per Funzioni Definite a Tratti:}
\begin{enumerate}
    \item Identifica tutti i punti di discontinuità
    \item Calcola i limiti destro e sinistro in ogni punto
    \item Applica il teorema di Dirichlet per determinare la convergenza
    \item Fai particolare attenzione ai punti agli estremi dell'intervallo di definizione
\end{enumerate}
\end{strategia}

\subsection{Tecniche Avanzate per Serie di Fourier}

\begin{info}
\textbf{Convergenza della Serie di Fourier:}
Per una funzione \(f\) periodica di periodo \(T\), la serie di Fourier converge:
\begin{itemize}
    \item \textbf{Puntualmente} in ogni punto dove \(f\) è continua
    \item \textbf{Al valor medio} nei punti di discontinuità: \(\frac{f(x^+) + f(x^-)}{2}\)
    \item \textbf{Totalmente} su intervalli dove \(f\) è continua
\end{itemize}
\end{info}

\begin{strategia}
\textbf{Calcolo Rapido dei Coefficienti:}
\begin{enumerate}
    \item Se \(f\) è \textbf{pari}: \(b_n = 0\) e \(a_n = \frac{4}{T}\int_0^{T/2} f(x)\cos\left(\frac{2\pi nx}{T}\right)dx\)
    \item Se \(f\) è \textbf{dispari}: \(a_n = 0\) e \(b_n = \frac{4}{T}\int_0^{T/2} f(x)\sin\left(\frac{2\pi nx}{T}\right)dx\)
    \item Per funzioni con \textbf{simmetrie speciali}, sfrutta le proprietà per ridurre i calcoli
\end{enumerate}
\end{strategia}

\begin{esempio}
\textbf{Funzione "Triangolare" dall'Esame 20 Giugno 2023:}
\[ g(x) = \begin{cases} 2\pi+x & x \in [-\pi,0) \\ 2\pi-x & x \in [0,\pi) \end{cases} \]

Osservazioni chiave:
\begin{enumerate}
    \item \(g\) è \textbf{pari}: \(g(-x) = g(x)\), quindi \(b_n = 0\)
    \item \(\hat{f}_0 = \frac{1}{2\pi}\int_{-\pi}^{\pi} g(x)dx = \frac{1}{\pi}\int_0^{\pi} (2\pi-x)dx = 2\pi - \frac{\pi}{2} = \frac{3\pi}{2}\)
    \item Per \(n \neq 0\): \(\hat{f}_n = \frac{1}{\pi}\int_0^{\pi} (2\pi-x)\cos(nx)dx\)
\end{enumerate}

Usando integrazione per parti:
\[ \hat{f}_n = \frac{2}{n^2}(\cos(n\pi) - 1) = \begin{cases} 
-\frac{4}{n^2} & n \text{ dispari} \\
0 & n \text{ pari}
\end{cases} \]
\end{esempio}

\begin{info}
\textbf{Identità di Parseval per Serie di Fourier:}
Se \(f\) ha periodo \(2\pi\):
\[ \frac{1}{\pi}\int_{-\pi}^{\pi} |f(x)|^2 dx = 2|a_0|^2 + \sum_{n=1}^{\infty}(|a_n|^2 + |b_n|^2) \]

Utile per verificare i calcoli e per calcolare somme di serie numeriche.
\end{info}
