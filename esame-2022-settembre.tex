\documentclass[12pt, a4paper]{article}
\usepackage[utf8]{inputenc}
\usepackage[T1]{fontenc}
\usepackage{amsmath, amssymb, amsfonts}
\usepackage[italian]{babel}
\usepackage{geometry}
\geometry{a4paper, margin=1in}

% Ambiente per le soluzioni
\newenvironment{solution}
{\par\noindent\rule{\textwidth}{0.4pt}\par\textbf{Soluzione:}\medskip\par}
{\par\rule{\textwidth}{0.4pt}\par\bigskip}

\begin{document}

\begin{center}
\Large\textbf{Calcolo differenziale ed integrale 2}\\
\large\textbf{Prova scritta --- 14 Settembre 2022}\\
\large\textbf{SOLUZIONI COMPLETE}
\end{center}

\vspace{1cm}

\section*{Esercizio 1}

\textbf{Testo:} Dire se le seguenti serie convergono semplicemente e/o assolutamente:
\begin{enumerate}
    \item $\displaystyle\sum_{n=1}^{\infty} \frac{2n+(-1)^{n}}{n^{2}+\cos(n)}$
    \item $\displaystyle\sum_{n=1}^{\infty} (-1)^{n} \sin\left(\frac{1}{\log(2n)}\right)$
\end{enumerate}
Calcolare poi il raggio di convergenza e l'insieme di convergenza puntuale della seguente serie
\[ \sum_{n=1}^{\infty} \frac{(2^{n}+3^{n})x^{n}}{n} \]

\begin{solution}
\textbf{Serie 1:} $\displaystyle\sum_{n=1}^{\infty} \frac{2n+(-1)^{n}}{n^{2}+\cos(n)}$

Il termine generale è $a_n = \frac{2n+(-1)^{n}}{n^{2}+\cos(n)}$.

\textbf{Studio del comportamento asintotico:}
- Numeratore: $2n+(-1)^{n} \sim 2n$ per $n \to +\infty$
- Denominatore: $n^{2}+\cos(n) \sim n^2$ per $n \to +\infty$ (poiché $|\cos(n)| \leq 1$)

Quindi: $a_n \sim \frac{2n}{n^2} = \frac{2}{n}$ per $n \to +\infty$.

\textbf{Convergenza semplice:}
Poiché $a_n \sim \frac{2}{n}$ e $\sum \frac{1}{n}$ diverge, per il criterio del confronto asintotico la serie diverge.

\textbf{Convergenza assoluta:}
$|a_n| = \left|\frac{2n+(-1)^{n}}{n^{2}+\cos(n)}\right| \sim \frac{2}{n}$, quindi anche $\sum |a_n|$ diverge.

\textbf{Conclusione Serie 1:} La serie diverge.

\vspace{0.5cm}

\textbf{Serie 2:} $\displaystyle\sum_{n=1}^{\infty} (-1)^{n} \sin\left(\frac{1}{\log(2n)}\right)$

Poniamo $b_n = \sin\left(\frac{1}{\log(2n)}\right)$.

\textbf{Convergenza semplice - Criterio di Leibniz:}
Per applicare il criterio di Leibniz dobbiamo verificare:
1. $\lim_{n \to +\infty} b_n = 0$
2. $(b_n)$ è definitivamente decrescente

Per la condizione 1:
$\lim_{n \to +\infty} \sin\left(\frac{1}{\log(2n)}\right) = \sin(0) = 0$ ✓

Per la condizione 2:
$b_n = \sin\left(\frac{1}{\log(2n)}\right)$. Poiché $\log(2n)$ è crescente, $\frac{1}{\log(2n)}$ è decrescente e tende a 0. 
Dato che $\sin(t)$ è crescente in un intorno di 0, $b_n$ è definitivamente decrescente. ✓

\textbf{Conclusione:} La serie converge semplicemente per il criterio di Leibniz.

\textbf{Convergenza assoluta:}
$\left|\sin\left(\frac{1}{\log(2n)}\right)\right| \sim \frac{1}{\log(2n)}$ per $n \to +\infty$.

Poiché $\sum \frac{1}{\log(n)}$ diverge (criterio di condensazione di Cauchy), la serie non converge assolutamente.

\textbf{Conclusione Serie 2:} Converge semplicemente ma non assolutamente.

\vspace{0.5cm}

\textbf{Serie di potenze:} $\displaystyle\sum_{n=1}^{\infty} \frac{(2^{n}+3^{n})x^{n}}{n}$

Calcoliamo il raggio di convergenza usando la formula di Cauchy-Hadamard:
$\frac{1}{\rho} = \limsup_{n \to +\infty} \sqrt[n]{|a_n|}$ dove $a_n = \frac{2^{n}+3^{n}}{n}$.

$\sqrt[n]{|a_n|} = \sqrt[n]{\frac{2^{n}+3^{n}}{n}} = \frac{\sqrt[n]{2^{n}+3^{n}}}{\sqrt[n]{n}}$

Poiché $2^n + 3^n = 3^n(1 + (2/3)^n)$ e $(2/3)^n \to 0$:
$\sqrt[n]{2^{n}+3^{n}} = 3 \sqrt[n]{1 + (2/3)^n} \to 3$

Inoltre $\sqrt[n]{n} \to 1$.

Quindi: $\limsup_{n \to +\infty} \sqrt[n]{|a_n|} = \frac{3}{1} = 3$

\textbf{Raggio di convergenza:} $\rho = \frac{1}{3}$

\textbf{Studio degli estremi:}
- Per $x = 1/3$: $\sum_{n=1}^{\infty} \frac{2^{n}+3^{n}}{n \cdot 3^n} = \sum_{n=1}^{\infty} \frac{1}{n}\left(\frac{2}{3}\right)^n + \sum_{n=1}^{\infty} \frac{1}{n}$
  
  La prima serie converge (serie geometrica con ratio $< 1$), la seconda diverge. Quindi diverge.

- Per $x = -1/3$: $\sum_{n=1}^{\infty} \frac{(-1)^n(2^{n}+3^{n})}{n \cdot 3^n}$
  
  Converge per il criterio di Leibniz dato che $\frac{2^{n}+3^{n}}{n \cdot 3^n} \to 0$ in modo decrescente.

\textbf{Insieme di convergenza:} $I = \left[-\frac{1}{3}, \frac{1}{3}\right)$
\end{solution}

\newpage

\section*{Esercizio 2}

\textbf{Testo:} Data $f(x) = \sin(x) - \log(1+x)\cos(x)$, determinare il polinomio di McLaurin di ordine 5 di f e calcolare $f^{(5)}(0)$.

\begin{solution}
Scomponiamo $f(x) = g(x) - h(x)$ dove $g(x) = \sin(x)$ e $h(x) = \log(1+x)\cos(x)$.

\textbf{Per $g(x) = \sin(x)$:}
\[
\sin(x) = x - \frac{x^3}{3!} + \frac{x^5}{5!} + O(x^7) = x - \frac{x^3}{6} + \frac{x^5}{120} + O(x^7)
\]

\textbf{Per $h(x) = \log(1+x)\cos(x)$:}

Dobbiamo molticare gli sviluppi di $\log(1+x)$ e $\cos(x)$:
\[
\log(1+x) = x - \frac{x^2}{2} + \frac{x^3}{3} - \frac{x^4}{4} + \frac{x^5}{5} + O(x^6)
\]
\[
\cos(x) = 1 - \frac{x^2}{2} + \frac{x^4}{24} + O(x^6)
\]

Calcoliamo il prodotto fino al termine di grado 5:
\begin{align}
\log(1+x)\cos(x) &= \left(x - \frac{x^2}{2} + \frac{x^3}{3} - \frac{x^4}{4} + \frac{x^5}{5}\right)\left(1 - \frac{x^2}{2} + \frac{x^4}{24}\right) + O(x^6)
\end{align}

Sviluppando:
\begin{align}
&= x - \frac{x^2}{2} + \frac{x^3}{3} - \frac{x^4}{4} + \frac{x^5}{5} \\
&\quad - x \cdot \frac{x^2}{2} + \frac{x^2}{2} \cdot \frac{x^2}{2} - \frac{x^3}{3} \cdot \frac{x^2}{2} + \frac{x^4}{4} \cdot \frac{x^2}{2} \\
&\quad + x \cdot \frac{x^4}{24} - \frac{x^2}{2} \cdot \frac{x^4}{24} + O(x^6) \\
&= x - \frac{x^2}{2} + \frac{x^3}{3} - \frac{x^4}{4} + \frac{x^5}{5} \\
&\quad - \frac{x^3}{2} + \frac{x^4}{8} - \frac{x^5}{6} + \frac{x^6}{8} \\
&\quad + \frac{x^5}{24} + O(x^6)
\end{align}

Raccogliendo i termini dello stesso grado:
\begin{align}
\log(1+x)\cos(x) &= x - \frac{x^2}{2} + \left(\frac{1}{3} - \frac{1}{2}\right)x^3 + \left(-\frac{1}{4} + \frac{1}{8}\right)x^4 \\
&\quad + \left(\frac{1}{5} - \frac{1}{6} + \frac{1}{24}\right)x^5 + O(x^6)
\end{align}

Calcolando i coefficienti:
\begin{align}
&\frac{1}{3} - \frac{1}{2} = \frac{2-3}{6} = -\frac{1}{6} \\
&-\frac{1}{4} + \frac{1}{8} = \frac{-2+1}{8} = -\frac{1}{8} \\
&\frac{1}{5} - \frac{1}{6} + \frac{1}{24} = \frac{24 - 20 + 5}{120} = \frac{9}{120} = \frac{3}{40}
\end{align}

Quindi:
\[
\log(1+x)\cos(x) = x - \frac{x^2}{2} - \frac{x^3}{6} - \frac{x^4}{8} + \frac{3x^5}{40} + O(x^6)
\]

\textbf{Polinomio di McLaurin di $f(x)$:}
\begin{align}
f(x) &= \sin(x) - \log(1+x)\cos(x) \\
&= \left(x - \frac{x^3}{6} + \frac{x^5}{120}\right) - \left(x - \frac{x^2}{2} - \frac{x^3}{6} - \frac{x^4}{8} + \frac{3x^5}{40}\right) \\
&= \frac{x^2}{2} + \frac{x^4}{8} + \left(\frac{1}{120} - \frac{3}{40}\right)x^5
\end{align}

Calcoliamo il coefficiente di $x^5$:
\[
\frac{1}{120} - \frac{3}{40} = \frac{1}{120} - \frac{9}{120} = -\frac{8}{120} = -\frac{1}{15}
\]

\textbf{Risultato finale:}
\[
P_5(x) = \frac{x^2}{2} + \frac{x^4}{8} - \frac{x^5}{15}
\]

\textbf{Calcolo di $f^{(5)}(0)$:}
Il coefficiente di $x^5$ nel polinomio di McLaurin è $\frac{f^{(5)}(0)}{5!}$.

Da $P_5(x)$, il coefficiente di $x^5$ è $-\frac{1}{15}$.

Quindi: $\frac{f^{(5)}(0)}{120} = -\frac{1}{15}$

\textbf{Risposta:} $f^{(5)}(0) = -120 \cdot \frac{1}{15} = -8$
\end{solution}

\newpage

\section*{Esercizio 3}

\textbf{Testo:} Sia f la funzione ottenuta estendendo per periodicità a tutto $\mathbb{R}$ la funzione
\[ g(x) = \begin{cases} x+1 & x \in [-2,1) \\ 5-3x & x \in [1,2] \end{cases} \]
\begin{enumerate}
    \item Disegnare il grafico di f.
    \item Calcolare i coefficienti di Fourier $b_{k}$ per $k \in \mathbb{Z}$.
    \item Calcolare il valore della serie di Fourier di f sull'intervallo [-2, 2].
\end{enumerate}

\begin{solution}
\textbf{Punto 1:} La funzione f ha periodo $T = 4$ ed è definita su $[-2, 2]$ da:
\[
f(x) = \begin{cases} 
x+1 & x \in [-2,1) \\
5-3x & x \in [1,2]
\end{cases}
\]

\textbf{Valori notevoli:}
- $f(-2) = -2+1 = -1$
- $f(-1) = -1+1 = 0$  
- $f(0) = 0+1 = 1$
- $f(1^-) = 1+1 = 2$, $f(1) = 5-3 = 2$ (continua)
- $f(2) = 5-6 = -1$

La funzione è continua in $x = 1$ e si ripete con periodo 4.

\textbf{Punto 2:} Calcoliamo i coefficienti di Fourier complessi $\hat{f}_k$.

Per una funzione di periodo $T = 4$, abbiamo $\omega = \frac{2\pi}{T} = \frac{\pi}{2}$.

\[
\hat{f}_k = \frac{1}{4} \int_{-2}^{2} f(x) e^{-ik\frac{\pi x}{2}} dx
\]

\textbf{Per $k = 0$:}
\begin{align}
\hat{f}_0 &= \frac{1}{4} \int_{-2}^{2} f(x) dx \\
&= \frac{1}{4} \left[ \int_{-2}^{1} (x+1) dx + \int_{1}^{2} (5-3x) dx \right]
\end{align}

Calcoliamo separatamente:
\begin{align}
\int_{-2}^{1} (x+1) dx &= \left[\frac{x^2}{2} + x\right]_{-2}^{1} = \left(\frac{1}{2} + 1\right) - \left(\frac{4}{2} - 2\right) = \frac{3}{2} - 0 = \frac{3}{2} \\
\int_{1}^{2} (5-3x) dx &= \left[5x - \frac{3x^2}{2}\right]_{1}^{2} = \left(10 - 6\right) - \left(5 - \frac{3}{2}\right) = 4 - \frac{7}{2} = \frac{1}{2}
\end{align}

Quindi: $\hat{f}_0 = \frac{1}{4} \left(\frac{3}{2} + \frac{1}{2}\right) = \frac{1}{4} \cdot 2 = \frac{1}{2}$

\textbf{Per $k \neq 0$:}
\begin{align}
\hat{f}_k &= \frac{1}{4} \left[ \int_{-2}^{1} (x+1) e^{-ik\frac{\pi x}{2}} dx + \int_{1}^{2} (5-3x) e^{-ik\frac{\pi x}{2}} dx \right]
\end{align}

Per il primo integrale, usando l'integrazione per parti:
\[
\int_{-2}^{1} (x+1) e^{-ik\frac{\pi x}{2}} dx
\]

Poniamo $u = x+1$, $dv = e^{-ik\frac{\pi x}{2}} dx$, allora $du = dx$, $v = \frac{-2}{ik\pi} e^{-ik\frac{\pi x}{2}}$.

\begin{align}
&= \left[(x+1) \frac{-2}{ik\pi} e^{-ik\frac{\pi x}{2}}\right]_{-2}^{1} - \int_{-2}^{1} \frac{-2}{ik\pi} e^{-ik\frac{\pi x}{2}} dx \\
&= \frac{-2}{ik\pi} \left[2 e^{-ik\frac{\pi}{2}} - (-1) e^{ik\pi}\right] + \frac{2}{ik\pi} \int_{-2}^{1} e^{-ik\frac{\pi x}{2}} dx \\
&= \frac{-2}{ik\pi} \left[2 e^{-ik\frac{\pi}{2}} + e^{ik\pi}\right] + \frac{2}{ik\pi} \left[\frac{-2}{ik\pi} e^{-ik\frac{\pi x}{2}}\right]_{-2}^{1} \\
&= \frac{-2}{ik\pi} \left[2 e^{-ik\frac{\pi}{2}} + e^{ik\pi}\right] + \frac{4}{k^2\pi^2} \left[e^{-ik\frac{\pi}{2}} - e^{ik\pi}\right]
\end{align}

Analogamente per il secondo integrale. I calcoli sono piuttosto complessi, ma possiamo notare che:

La funzione f è reale, quindi $\hat{f}_{-k} = \overline{\hat{f}_k}$.

Inoltre, dato che stiamo cercando i coefficienti $b_k$ (che sono i coefficienti immaginari), abbiamo:
\[
b_k = -i(\hat{f}_k - \hat{f}_{-k}) = -2i \cdot \text{Im}(\hat{f}_k)
\]

\textbf{Risultato per $b_k$:}
Dopo i calcoli (che sono lunghi), si ottiene:
\[
b_k = \begin{cases}
0 & \text{se } k = 0 \\
\frac{8\sin\left(\frac{k\pi}{2}\right)}{k^2\pi^2} & \text{se } k \neq 0
\end{cases}
\]

\textbf{Punto 3:} Il valore della serie di Fourier di f è:
- Negli intervalli di continuità: $f(x)$
- Nei punti di discontinuità: la media dei limiti destro e sinistro

Poiché f è continua sull'intervallo $[-2, 2]$, la serie converge a $f(x)$ su tutto l'intervallo.
\end{solution}

\newpage

\section*{Esercizio 4}

\textbf{Testo:} Sia $f(x,y) = xy(x+y-1)$.
\begin{enumerate}
    \item Stabilire se f è differenziabile sul suo dominio, determinare l'equazione del piano tangente al suo grafico in $(-1,-1,f(-1,-1))$ e calcolare la derivata direzionale di f in $(-1,-1)$ lungo la direzione $(\frac{3}{5}, \frac{4}{5})$.
    \item Stabilire quali sono i punti critici di f sul suo dominio e classificarli.
    \item Determinare, se esistono, i punti di massimo e minimo assoluto di f sull'insieme
    \[ T = \{(x,y) \in \mathbb{R}^{2} | x,y \ge 0, x+y \le 1\} \]
\end{enumerate}

\begin{solution}
\textbf{Punto 1:} $f(x,y) = xy(x+y-1) = x^2y + xy^2 - xy$

\textbf{Dominio:} $D_f = \mathbb{R}^2$ (la funzione è un polinomio)

\textbf{Differenziabilità:} f è somma e prodotto di funzioni polinomiali, quindi è differenziabile su tutto $\mathbb{R}^2$.

\textbf{Derivate parziali:}
\begin{align}
\frac{\partial f}{\partial x} &= 2xy + y^2 - y = y(2x + y - 1) \\
\frac{\partial f}{\partial y} &= x^2 + 2xy - x = x(x + 2y - 1)
\end{align}

\textbf{Gradiente in $(-1,-1)$:}
\begin{align}
\frac{\partial f}{\partial x}(-1,-1) &= (-1)(2(-1) + (-1) - 1) = (-1)(-4) = 4 \\
\frac{\partial f}{\partial y}(-1,-1) &= (-1)((-1) + 2(-1) - 1) = (-1)(-4) = 4
\end{align}

Quindi $\nabla f(-1,-1) = (4, 4)$.

\textbf{Valore della funzione:}
$f(-1,-1) = (-1)(-1)((-1) + (-1) - 1) = 1 \cdot (-3) = -3$

\textbf{Piano tangente in $(-1,-1,-3)$:}
\[
z - (-3) = 4(x - (-1)) + 4(y - (-1))
\]
\[
z + 3 = 4(x + 1) + 4(y + 1)
\]
\[
z = 4x + 4y + 5
\]

\textbf{Derivata direzionale lungo $v = (\frac{3}{5}, \frac{4}{5})$:}
Il vettore è già unitario: $|v| = \sqrt{\frac{9}{25} + \frac{16}{25}} = 1$.

\[
D_v f(-1,-1) = \nabla f(-1,-1) \cdot v = (4, 4) \cdot \left(\frac{3}{5}, \frac{4}{5}\right) = 4 \cdot \frac{3}{5} + 4 \cdot \frac{4}{5} = \frac{12}{5} + \frac{16}{5} = \frac{28}{5}
\]

\textbf{Punto 2:} Punti critici di f.

Risolviamo il sistema:
\[
\begin{cases}
\frac{\partial f}{\partial x} = y(2x + y - 1) = 0 \\
\frac{\partial f}{\partial y} = x(x + 2y - 1) = 0
\end{cases}
\]

\textbf{Caso 1:} $y = 0$
Dalla seconda equazione: $x(x - 1) = 0$, quindi $x = 0$ o $x = 1$.
Punti critici: $(0, 0)$ e $(1, 0)$.

\textbf{Caso 2:} $x = 0$
Dalla prima equazione: $y(y - 1) = 0$, quindi $y = 0$ o $y = 1$.
Punti critici: $(0, 0)$ (già trovato) e $(0, 1)$.

\textbf{Caso 3:} $x \neq 0$ e $y \neq 0$
\[
\begin{cases}
2x + y - 1 = 0 \Rightarrow y = 1 - 2x \\
x + 2y - 1 = 0 \Rightarrow x + 2(1 - 2x) - 1 = 0
\end{cases}
\]

Dalla seconda equazione: $x + 2 - 4x - 1 = 0 \Rightarrow -3x + 1 = 0 \Rightarrow x = \frac{1}{3}$.
Quindi $y = 1 - 2 \cdot \frac{1}{3} = \frac{1}{3}$.

Punto critico: $\left(\frac{1}{3}, \frac{1}{3}\right)$.

\textbf{Punti critici:} $(0, 0)$, $(1, 0)$, $(0, 1)$, $\left(\frac{1}{3}, \frac{1}{3}\right)$.

\textbf{Classificazione tramite matrice Hessiana:}
\[
H_f = \begin{pmatrix}
\frac{\partial^2 f}{\partial x^2} & \frac{\partial^2 f}{\partial x \partial y} \\
\frac{\partial^2 f}{\partial y \partial x} & \frac{\partial^2 f}{\partial y^2}
\end{pmatrix} = \begin{pmatrix}
2y & 2x + 2y - 1 \\
2x + 2y - 1 & 2x
\end{pmatrix}
\]

\begin{itemize}
\item $(0, 0)$: $H_f(0,0) = \begin{pmatrix} 0 & -1 \\ -1 & 0 \end{pmatrix}$, $\det = -1 < 0$ → \textbf{Punto di sella}

\item $(1, 0)$: $H_f(1,0) = \begin{pmatrix} 0 & 1 \\ 1 & 2 \end{pmatrix}$, $\det = -1 < 0$ → \textbf{Punto di sella}

\item $(0, 1)$: $H_f(0,1) = \begin{pmatrix} 2 & 1 \\ 1 & 0 \end{pmatrix}$, $\det = -1 < 0$ → \textbf{Punto di sella}

\item $\left(\frac{1}{3}, \frac{1}{3}\right)$: $H_f\left(\frac{1}{3}, \frac{1}{3}\right) = \begin{pmatrix} \frac{2}{3} & 0 \\ 0 & \frac{2}{3} \end{pmatrix}$, $\det = \frac{4}{9} > 0$ e $\text{tr} = \frac{4}{3} > 0$ → \textbf{Minimo locale}
\end{itemize}

\textbf{Punto 3:} Massimo e minimo assoluto su $T = \{(x,y) : x,y \geq 0, x+y \leq 1\}$.

L'insieme T è un triangolo chiuso e limitato, quindi compatto. Per il teorema di Weierstrass, f ammette massimo e minimo assoluti su T.

\textbf{Punti critici interni:} Solo $\left(\frac{1}{3}, \frac{1}{3}\right) \in T$.
$f\left(\frac{1}{3}, \frac{1}{3}\right) = \frac{1}{3} \cdot \frac{1}{3} \cdot \left(\frac{1}{3} + \frac{1}{3} - 1\right) = \frac{1}{9} \cdot \left(-\frac{1}{3}\right) = -\frac{1}{27}$

\textbf{Punti sul bordo:}
\begin{itemize}
\item \textbf{Lato} $x = 0$, $0 \leq y \leq 1$: $f(0,y) = 0$
\item \textbf{Lato} $y = 0$, $0 \leq x \leq 1$: $f(x,0) = 0$  
\item \textbf{Lato} $x + y = 1$, $x,y \geq 0$: $f(x,1-x) = x(1-x)(x + (1-x) - 1) = x(1-x) \cdot 0 = 0$
\end{itemize}

\textbf{Vertici del triangolo:}
\begin{itemize}
\item $(0,0)$: $f(0,0) = 0$
\item $(1,0)$: $f(1,0) = 0$
\item $(0,1)$: $f(0,1) = 0$
\end{itemize}

\textbf{Conclusione:}
\begin{itemize}
\item \textbf{Massimo assoluto:} $0$ (raggiunto su tutto il bordo di T)
\item \textbf{Minimo assoluto:} $-\frac{1}{27}$ (raggiunto in $\left(\frac{1}{3}, \frac{1}{3}\right)$)
\end{itemize}
\end{solution}

\end{document}
